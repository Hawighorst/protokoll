\subsection{Daten und Analyse}
\begin{table}[h]
	Die in den Abbildungen \ref{fig:Kreiselunten},\ref{fig:Kreiselmitte} und \ref{fig:Kreiseloben} zu sehenden Steigungen der linearen Anpassung ist in \ref{tab:Kreisel} zusehen.
	Man erkennt in den Abbildungen das einige Werte zum teil sehr Stark von der Anpassung abweichen, dies ist vermutlich auf Fehler bei der Durchführung zurückzuführen.
	\caption{Zu sehen sind die Werte für Die Steigung die Kraft für die drei unterschiedlichen Positionen sowie die Masse der Kugel, Radius der Kugel sowie die Länge l.}
	\begin{tabular}{|c|c|c|c|c|c|}
		\hline
		Position & $\frac{\Delta T_p}{\Delta \omega}$ & Kraft& Radius & Masse &Länge l\\
		\hline
		Unten 1 & \SI{5,13+-0.09e-2}{1/s^2} & \SI{0,144+-0,005}{N} & \SI{25,39+-0,01}{mm} & \SI{512,240+-0,003}{g}& \SI{84.82+-0,02}{mm}\\
		\hline
		Mitte 2& \SI{4,31+-0,02e-2}{1/s^2}& \SI{0,166+-0,005}{N} & \SI{25,39+-0,01}{mm} & \SI{512,240+-0,003}{g}&\SI{84.82+-0,02}{mm}\\
		\hline
		Oben 3 & \SI{3,59+-0,03e-2}{1/s^2}& \SI{0,220+-0,005}{N} & \SI{25,39+-0,01}{mm} & \SI{512,240+-0,003}{g}&\SI{84.82+-0,02}{mm}\\
		\hline
		\end{tabular}
	\label{tab:Kreisel}
\end{table}
\begin{figure}[h]
\centering
\includegraphics[width=0.7\linewidth]{res/sproHzunten}
\caption{{Aufgetragen ist die die Präzessionzeit $T_p$ gegen die Kreisfrequenz $\omega$ bei Positionierung des Zusatzgewichtes bei Position 1.}
	\label{fig:Kreiselunten}
\end{figure}
\begin{figure}[h]
	\centering
	\includegraphics[width=0.7\linewidth]{res/sproHzmitte}
	\caption{Aufgetragen ist die die Präzessionzeit $T_p$ gegen die Kreisfrequenz $\omega$ bei Positionierung des Zusatzgewichtes bei Position 2.}
	\label{fig:Kreiselmitte}
\end{figure}

\begin{figure}[h]
	\centering
	\includegraphics[width=0.7\linewidth]{res/sproHzoben}
	\caption{{Aufgetragen ist die die Präzessionzeit $T_p$ gegen die Kreisfrequenz $\omega$ bei Positionierung des Zusatzgewichtes bei Position 3.}
		\label{fig:Kreiseloben}
	\end{figure}
Im nächsten Schritt wurde das Produkt der Länge l un der Kräfte F gegen den Kehrwert der Steigung aus den Abbildungen \ref{fig:Kreiselunten},\ref{fig:Kreiselmitte} und \ref{fig:Kreiseloben} aufgetragen. 
Daraus resultierte die Abbildung \ref{Kegel}.
Man erkennt das die lineare Anpassung die drei Messwerte nicht besonders gut Approximiert da keiner der Punkte auf der Geraden liegt. Dies ist wie schon vorher erwähnt auf Fehler bei der Durchführung zurückzuführen.
Über den in Kapitel 
\begin{figure}[h]
	\centering
	\includegraphics[width=0.7\linewidth]{res/wtgegenlF}
	\caption{Aufgetragen ist das Produkt l mal F gegen $\frac{\Delta \omega}_{\Delta T_p}$}
	\label{fig:Kreisel}
\end{figure}
