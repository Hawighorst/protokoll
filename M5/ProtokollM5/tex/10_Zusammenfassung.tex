%Zusammenfassung in unter 200 Wörtern

\section{Zusammenfassung}

Der Versuchtag bestand aus zwei Experimenten welche die Rotation starrer Körper betrachten, zunächst wurde das Fallverhalten des Maxwellsche Fallrad, ähnlich einem Jo-Jo, untersucht und anschließend die Präzessionsbewegung eines Kreisels. 



%Fallrad


Das Maxwellsche Fallrad eignet sich zur Untersuchung von gleichmäßig beschleunigten Bewegungen, da die potentielle Energie in Translation und Rotation umgewandelt wird somit hat das Rad eine geringere Geschwindigkeit und die Bewegung lässt sich ohne aufwendige Messinstrumente beobachten, da die Fallzeiten groß genug waren um sie mit einer Herkömmlichen Stoppuhr zu messen.
Aus Abmessungen und Gewicht wurde das Trägheitsmoment bezüglich der Symmetrieachse $J_s=\SI{0.003702 \pm 0.000008}{kg\cdot m^2}$ bestimmt, anschließend wurde aus den Fallzeiten die effektive Beschleunigung  $g^*=\SI{0.0410 \pm  0.0019}{m \per s \squared}$ bestimmt. Abschließend wurde mit dem Steinerschen Satz auf den Abrollradius $R=\SI{0.00460\pm 0.00011}{m}$ geschlossen und mit dem gemessenen Abrollradius $R_{geometrisch}=\SI{0.00455 \pm 0.00004}{m}$ verglichen. Der geometrisch bestimmte Wert bestätigt die vorherige Messung. 






 

%Kreisel












