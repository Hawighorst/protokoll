%Zusammenfassung in unter 200 Wörtern

\section{Zusammenfassung}\label{kap:Zusammenfassung}

Der Versuchtag bestand aus zwei Experimenten, welche die Rotation starrer Körper betrachten, zunächst wurde das Fallverhalten des Maxwellsche Fallrad, ähnlich einem Jo-Jo, untersucht und anschließend die Präzessionsbewegung eines Kreisels. 

%Fallrad


Bei dem Maxwellschen Fallrad wurde eine gleichmäßig beschleunigte Translation und Rotation beobachtet.
Aus Abmessungen und Gewicht wurde das Trägheitsmoment bezüglich der Symmetrieachse $J_s=\SI{0.003702 \pm 0.000008}{kg\cdot m^2}$ bestimmt, anschließend wurde aus den Fallzeiten die effektive Beschleunigung  $g^*=\SI{0.0410 \pm  0.0019}{m \per s \squared}$ bestimmt. Abschließend wurde mit dem Steinerschen Satz auf den Abrollradius $R=\SI{0.00460\pm 0.00011}{m}$ geschlossen und mit dem gemessenen Abrollradius $R_{geometrisch}=\SI{0.00455 \pm 0.00004}{m}$ verglichen. Der geometrisch bestimmte Wert bestätigt die vorherige Messung. 






%Kreisel

Bei dem Kreisel wurde die Präzessionszeit $T_p$ bei annähernd konstanter Winkelgeschwindigkeit $\omega$ gemessen. Bei dem untersuchten Kreisel handelte es sich um einen schweren, symmetrischen Kreisel.
Es sollte das Trägheitsmoment $J$ des Kreisels bestimmt werden. Einerseits experimentell über den Zusammenhang zwischen $\frac{\Delta \omega}{\Delta T_p}$ und dem Produkt aus der Kraft~F und dem Abstand~l zwischen dem Unterstützungspunkt und dem Angriffspunkt des Kraftmessers vgl. Abb. \ref{fig:Kreisel} (im folgendem $J_{exp.}$ genannt). Sowie aus der Masse und dem Radius der Kugel zuzüglich einem gegebenen Trägheitsmoment des Stabes mit dem Zusatzgewicht im folgendem $J_{theo}$ genannt. Da sich $J_{theo.}=\SI{1,3356+-0,0004e-4}{kg \cdot m^2}$ und $J_{exp.}=\SI{1,0194+-0,0319e-4}{kg \cdot m^2}$ deutlich voneinander Unterscheiden ist zu vermuten, dass beim Experimentieren Fehler unterlaufen sind.  
 

%vermutlich darauf zurückzuführen







