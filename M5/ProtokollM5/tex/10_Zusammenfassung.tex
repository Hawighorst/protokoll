%Zusammenfassung in unter 200 Wörtern

\section{Zusammenfassung}

Der Versuchtag bestand aus zwei Experimenten welche die Rotation starrer Körper betrachten, zunächst wurde das Fallverhalten des Maxwellsche Fallrad, ähnlich einem Jo-Jo, untersucht und anschließend die Präzessionsbewegung eines Kreisels. 



%Fallrad


Das Maxwellsche Fallrad eignet sich zur Untersuchung von gleichmäßig beschleunigten Bewegungen, da die potentielle Energie in Translation und Rotation umgewandelt wird somit hat das Rad eine geringere Geschwindigkeit und die Bewegung lässt sich ohne aufwendige Messinstrumente beobachten, da die Fallzeiten groß genug waren um sie mit einer Herkömmlichen Stoppuhr zu messen.
Aus Abmessungen und Gewicht wurde das Trägheitsmoment bezüglich der Symmetrieachse $J_s=\SI{0.003702 \pm 0.000008}{kg\cdot m^2}$ bestimmt, anschließend wurde aus den Fallzeiten die effektive Beschleunigung  $g^*=\SI{0.0410 \pm  0.0019}{m \per s \squared}$ bestimmt. Abschließend wurde mit dem Steinerschen Satz auf den Abrollradius $R=\SI{0.00460\pm 0.00011}{m}$ geschlossen und mit dem gemessenen Abrollradius $R_{geometrisch}=\SI{0.00455 \pm 0.00004}{m}$ verglichen. Der geometrisch bestimmte Wert bestätigt die vorherige Messung. 






 

%Kreisel

Im zweiten Experiment wurde die Präzessionszeit $T_p$ eines Kreisels bei einer annähernd Konstanten Winkelgeschwindigkeit $\omega$. Bei dem Untersuchten Kreisel handelte es sich um einen Schweren Symmetrischen Kreisel.
Durch das Experiment sollte das Trägheitsmoment J des Kreisels Bestimmt werden. Einmal experimentell über den Zusammenhang zwischen $\frac{\Delta \omega}{\Delta T_p}$ und dem Produkt aus der Kraft F und dem Abstand l zwischen dem Unterstützungspunkt und dem Angriffspunkt des Kraftmessers vgl. Abb. \ref{fig:Kreisel}(im folgendem $J_{exp.}$ genannt). Und einmal aus der Masse und dem Radius der Kugel sowie einem gegebenen Trägheitsmoment des Stabes mit dem Zusatzgewicht im folgendem $J_{theo}$ genannt. Da sich $J_{theo.}=\SI{1,3356+-0,0004e-4}{kg \cdot m^2}$ und $J_{exp.}=\SI{1,0194+-0,0319e-4}{kg \cdot m^2}$ deutlich von einander Unterscheiden ist vermutlich darauf zurückzuführen das beim experimentieren Fehler unterlaufen sind.  
 









