\subsection{Unsicherheiten Kreisel}
Bei der Durchführung des Experimentes traten an einigen Stellen sowohl bei der Durchführung als auch beim Rechnen Unsicherheiten auf die im folgendem näher erläutert werden.
Die in den Abbildungen \ref{fig:Kreiselunten},\ref{fig:Kreiselmitte} und \ref{fig:Kreiseloben} zu sehenden Fehlerbalken ergeben sich einmal für die Präzessionszeit $T_p$ durch die Gleichungen \ref{eq:sur},\ref{eq:sud} und \ref{eq:kombsu}.
Die Rechteckverteilung ist durch die Ableseungenauigkeit der Stoppuhr mit a=\SI{0.01}{s} und die Dreiecksverteilung ergibt sich durch die Reaktionszeit wobei a=\SI{0,4}{s} angenommen wurde. Dieses Ergebnis wurde dann durch die Anzahl an Umdrehungen der Stange des Kreisel Geteilt.
Für die Winkelgeschwindigkeit ergibt sich der Fehler aus dem Fehler der Frequenz f multipliziert mit $2\pi$. Mithilfe einer Dreiecksverteilung konnte die Unsicherheit der Frequenz abgeschätzt werden, da die Werte der Frequenz von einer analogen Skala abgelesen wurden. Ein weiterer Fehler der in Betracht gezogen werden muss ist, dass die Markierung auf der Kugeloberfläche nie vollständig Stillstand sondern wanderte. Da die Markierung jedoch immer wieder an die gleiche stelle zurück wanderte und nie nur in eine Richtung wurde im folgendem davon ausgegangen das sich die Effekte gegenseitig ausgeglichen haben.
Bei den Fehlerbalken in Abbildung \ref{fig:Kreisel} ergeben sich die Fehlerbalken einmal für den Kehrwert der Steigungen durch den Prozentualen Fehler der Steigung Multipliziert mit dem Kehrwert der Steigungen aus den Abbildungen \ref{fig:Kreiselunten},\ref{fig:Kreiselmitte} und \ref{fig:Kreiseloben} (Diese Fehler wurden dem Fit Programm entnommen). 
Die Fehlerbalken des  l $\cdot$ F Produktes ergeben sich nach Gleichung \ref{eq:suw} mit den im folgenden noch aufgeführten Unsicherheiten für l und F.
Die Messungen des Radius und der Länge der Stange l wurden mit einem Messschieber mit Nonius gemessen. Bei der Messung viel auf das meistens jeweils  zwei verschiedene Messergebnisse gleich wahrscheinlich  waren. Aus diesem Grund 
sind die Messergebnisse mit einer Unsicherheit nach Gleichung \ref{eq:sud} mit a= \SI{0,12}{mm} versehen, wobei bei der Unsicherheit des Radius darauf zu achten war das der Durchmesser bestimmt wurde wesshalb der fehler durch zwei dividiert werden musste. Da die Kraft mit einem Kraftmesser mit analoger anzeige bestimmt, somit ergibt sich die Unsicherheit nach \ref{eq:sud} mit a=\SI{0,2}{N}. Für die Masse  ergab sich die Unsicherheit durch die Messung des Gewichts mithilfe einer Digitalen Waage. Also ergab sie sich nach Gleichung \ref{eq:sur} mit a=\SI{0,01}{g}. 
Die Unsicherheit von $J_{theo.}$ ergibt sich nach Gleichung \ref{eq:suw} angewandt auf die Gleichung \ref{eq:J} mit den nach den oben genannten Gleichungen für Unsicherheiten der Masse und des Radius.
Um die Unsicherheit von $J_{exp.}$ zu erhalten wurde die Unsicherheit der Steigung (Die Steigung der Geraden aus Abb. \ref{fig:Kreisel}) dividiert durch zwei $\pi$. 