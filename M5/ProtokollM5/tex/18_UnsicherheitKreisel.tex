\subsection{Unsicherheiten des Kreisel}
Bei dem Experiment traten an einigen Stellen, sowohl bei der Durchführung als auch beim Rechnen Unsicherheiten auf, die im Folgendem näher erläutert werden.
Die in den Abbildungen \ref{fig:Kreiselunten}, \ref{fig:Kreiselmitte} und \ref{fig:Kreiseloben} eingezeichneten Fehlerbalken ergeben sich für die Präzessionszeit $T_p$ durch die Gleichungen \ref{eq:sur}, \ref{eq:sud} und \ref{eq:kombsu}.
Die Rechteckverteilung (\ref{eq:sur}) ist durch die Ableseungenauigkeit der Stoppuhr mit $a=\SI{0.01}{s}$ und die Dreiecksverteilung (\ref{eq:sud}) ergibt sich durch die Reaktionszeit wobei $a=\SI{0,4}{s}$ angenommen wurde. Dieses Ergebnis wurde im Anschluss durch die Anzahl der Umdrehungen der Stange des Kreisel geteilt.\\

Für die Kreisfrequenz ergibt sich eine Unsicherheit von $2 \pi \Delta f$ bedingt durch die analoge Skala des Stroboskops. Sie wurde durch eine Dreiecksverteilung abgeschätzt. %Mithilfe einer Dreiecksverteilung konnte die Unsicherheit der Frequenz abgeschätzt werden, da die Werte der Frequenz von einer analogen Skala abgelesen wurden. 
Ein weiterer Fehler der in Betracht gezogen werden muss ist, dass die Markierung auf der Kugeloberfläche nie vollständig stillstand, sondern wanderte. Da diese Abweichungen jedoch in gleichem Maße in beide Richtungen Abwichen, ist davon auszugehen dass diese nicht durch größere Unsicherheiten zu berücksichtigen sind.
%Da die Markierung jedoch im Mittel über die Messperiode keine Bewegung ausführte, wirde davon ausgegegangen, dass die Unsicherheit in der Frequenz keinen Einfluss auf die Präzessionszeit hat. 
%Eine Abschätzung liefert des weiteren, dass selbst eine Umdrehung der Markierung in $\SI{10}{s}$ bei einer Frequenz von $\SI{350}{Hz}$ unterhalb der Größenordnung der anderen Fehler ist. 
%immer wieder an die gleiche Stelle zurück wanderte und nie nur in eine Richtung wurde im folgendem davon ausgegangen das sich die Effekte gegenseitig ausgeglichen haben.
Bei den Fehlerbalken in Abbildung \ref{fig:Kreisel} ergeben sich die Fehlerbalken einmal für den Kehrwert der Steigungen durch den prozentualen Fehler der Steigung multipliziert mit dem Kehrwert der Steigungen aus den Abbildungen \ref{fig:Kreiselunten}, \ref{fig:Kreiselmitte} und \ref{fig:Kreiseloben}.Die Unsicherheiten der Steigungen wurden dem Fit Programm entnommen. 
Die Fehlerbalken des  l $\cdot$ F Produktes ergeben sich nach Gleichung \ref{eq:suw} mit den Unsicherheiten für $l$ und $F$ nach \cref{tab:Kreisel}.\\

Des Radius $R$ und die Länge der Stange $l$ wurden mithilfe eines Messschiebers mit Nonius gemessen. Bei der Messung fiel auf, dass meistens jeweils  zwei Messergebnisse gleich wahrscheinlich  waren. Aus diesem Grund sind die Messergebnisse mit einer Unsicherheit nach Gleichung \ref{eq:sud} mit $a=\SI{0,12}{mm}$ versehen, wobei bei der Unsicherheit des Radius darauf zu achten war das der Durchmesser bestimmt wurde, weshalb der Fehler halbiert werden musste. Da die Kraft durch einem Kraftmesser mit analoger Anzeige bestimmt wurde, ergibt sich die Unsicherheit bezüglich $F$ nach \ref{eq:sud} mit $a=\SI{0,2}{N}$. Für die Masse  ergab sich die Unsicherheit durch die Messung des Gewichts mithilfe einer digitalen Waage. Also ergab sie sich nach Gleichung \ref{eq:sur} mit $a=\SI{0,01}{g}$. 
Die Unsicherheit von $J_{theo.}$ ergibt sich nach Gleichung \ref{eq:suw} angewandt auf die Gleichung \ref{eq:J} mit den nach den obigen Gleichungen für Unsicherheiten der Masse und des Radius.
Die Unsicherheit des Trägheitsmomentes $J_{exp.}$ folgt aus der Unsicherheit der Steigung $S$ der Geraden aus \cref{fig:Kreisel} mit $\Delta J_{exp.}=\frac{\Delta S}{2  \pi}$.
 %zu erhalten wurde die Unsicherheit der Steigung der Geraden aus Abb. \ref{fig:Kreisel} dividiert durch zwei $\pi$. 

