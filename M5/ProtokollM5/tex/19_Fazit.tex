%19
\FloatBarrier
\section{Schlussfolgerung}

Bei dem Maxwellschen Fallrad wurde zunächst aus Abmessungen und Gewicht das Trägheitsmoment bezüglich der Symmetrieachse $J_s=\SI{0.003702 \pm 0.000008}{kg\cdot m^2}$ errechnet, anschließend wurde aus den Fallzeiten die effektive Beschleunigung  $g^*=\SI{0.0410 \pm  0.0019}{m \per s \squared}$ bestimmt. Abschließend wurde mit dem Steinerschen Satz auf den Abrollradius $R=\SI{0.00460\pm 0.00011}{m}$ geschlossen und mit dem gemessenen Abrollradius $R_{geometrisch}=\SI{0.00455 \pm 0.00004}{m}$ verglichen. Der geometrisch bestimmte Wert bestätigt zum einen die vorherige Messung ist jedoch einfacher und genauer direkt zu bestimmen. Des weiteren plausibilisiert die Übereinstimmung der Abrollradien die Werte für effektive Beschleunigung und Trägheitsmoment. Somit bestätigen die Messwerte die theoretischen Annahmen.

Bei dem Kreisel wurden für das Trägheitsmoment aus den Abmessungen und der Msse  $J_{theo.}=\SI{1,3356+-0,0004e-4}{kg \cdot m^2}$ und aus dem Kreiselverhalten $J_{exp.}=\SI{1,0194+-0,0319e-4}{kg \cdot m^2}$ berechnet. Die Werte unterscheiden sich um ca. 30\%.
Dies ist entweder auf einen groben Rechenfehler zurückzuführen oder auf Fehler bei der Durchführung des Experimentes. Da auch nach mehrfacher Durchsicht kein Fehler gefunden werden konnte, ist es jedoch wahrscheinlicher das der Fehler bei der Durchführung des Experimentes zu suchen ist. Bei der Auswertung der Messwerte fiel auf das der Kehrwert der Steigung aus \cref{fig:Kreiselmitte} nach unten am Stärksten von der linearen Anpassung abweicht und das obwohl sich die Gemessenen Werte für die Präzessionszeit hier am besten Linearisieren ließen. Dies lässt einen systematischen Fehler zum Beispiel im Aufbau oder der Durchführung 
vermuten. Um verwertbare Ergebnisse zu erhalten müsste dieser Teil des Experimentes wiederholt werden.








