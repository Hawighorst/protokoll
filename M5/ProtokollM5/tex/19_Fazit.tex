%19
%\clearpage
\section{Schlussfolgerung}

Bei dem Maxwellschen Fallrad wurde zunächst aus Abmessungen und Gewicht das Trägheitsmoment bezüglich der Symmetrieachse $J_s=\SI{0.003702 \pm 0.000008}{kg\cdot m^2}$ errechnet, anschließend wurde aus den Fallzeiten die effektive Beschleunigung  $g^*=\SI{0.0410 \pm  0.0019}{m \per s \squared}$ bestimmt. Abschließend wurde mit dem Steinerschen Satz auf den Abrollradius $R=\SI{0.00460\pm 0.00011}{m}$ geschlossen und mit dem gemessenen Abrollradius $R_{geometrisch}=\SI{0.00455 \pm 0.00004}{m}$ verglichen. Der geometrisch bestimmte Wert bestätigt zum einen die vorherige Messung ist jedoch einfacher und genauer direkt zu bestimmen. Des weiteren plausibilisiert die Übereinstimmung der Abrollradien die Werte für effektive Beschleunigung und Trägheitsmoment. Somit bestätigen die Messwerte die theoretischen Annahmen.








