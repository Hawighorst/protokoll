\section{Methoden}
Wie aus \cref{theorie} deutlich wird, lässt sich die Fallbeschleunigung aus Kreisradius und Schwingungsdauer berechnen. Zunächst wurden daher fünfmal die Zeit für 20 Pendelschwingungen gemessen, um sowohl das Risiko des Verzählens als auch die Messungenauigkeiten durch das manuelle Zeitnehmen zu minimieren. In einer zweiten Messreihe wurden jeweils weniger Pendelschläge (30) gezählt, jedoch wurde verschiedene Pendellänge ausgewertet.
Die Längen wurden mit einem Maßband bestimmt, gemessen wurde von der Aufhängung bis zu der Oberseite der Metallkugel anschließend wurde der Kugelradius (\SI{15}{mm}) addiert, da die für die Schwingungsdauer relevante Länge der Abstand zwischen Aufhängung und Schwerpunkt ist.
Die Schwingungsdauer wurde mit einer Stoppuhr bestimmt. Start und Endpunkt wurden, mithilfe des Kugelschattens, eindeutig auf dem Tisch gekennzeichnet. Als Messpunkt wurde die Stelle maximaler Geschwindigkeit gewählt, um den Einfluss einer Unsicherheit des Ortes auf die Schwingungsdauer zu minimieren. 
