\subsection{Analyse von Schaltkreis c)}
In diesem Teil des Protokolls wird ein der Schaltkreis mit der Spule die in Reihe mit einem Kondensator geschaltet wurde.
Dazu wurde zum einem der Innenwiderstand $R_i$ und die Induktivität $L$ der Spule aus der Auswertung aus Kapitel \ref{kap:Spule}.
Im folgenden wird mithilfe von Abbildung \ref{label}
und den folgenden Gleichungen:
\begin{align}
C=\frac{1}{\omega (\omega L+\sqrt{Z^2-R^2})}\\	
\phi = \arccos\left(\frac{\bar{P}}{U_{eff.}I_{eff.}}\right)\\
	|Z|=\frac{U_{eff.}}{I_{eff.}}\\	
\end{align}
die Kapazität $C$ des Kondensators, der Betrag des Scheinwiderstandes und der Betrag der Phase berechnet. Und das Ergebnis für den Kondesator mit dem vom Kondensator abgelesenen Wert für die Kapazität verglichen.
Der Kondensator hatte nach Hersteller angaben eine Kapazität von \SI{60+-3.46}{µF}.
Dieser Wert liegt in der $1\sigma$-Umgebung des errechneten Wertes: \SI{57.4+-3.8}{µF} und stimmt somit gut mit dem Theoretischen Wert überein.
Um die Unsicherheiten der Werte zu erhalten wurde einmal für den Herstellerwert die von diesem angegebene Unsicherheit von $10\%$ nach Gleichung
\ref{eq:su3} abgeschätzt und die Unsicherheit des Messwertes ergibt nach Gleichung \ref{eq:uKond}.
Die anderen Ergebnisse Lauten:
\begin{align}
	\phi=SI{0.9989+-0.0006}{rad}\\
	|Z|=SI{43.1+-3.1}{\ohm}
\end{align}
