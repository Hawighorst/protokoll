\subsection{Analyse von Schaltkreis c)}
In diesem Teil des Protokolls wird ein der Schaltkreis mit der Spule die in Reihe mit einem Kondensator geschaltet wurde.
Dazu wurde zum einem der Innenwiderstand $R_i$ und die Induktivität $L$ der Spule aus der Auswertung aus Kapitel \ref{kap:Spule}.
Im folgenden wird mithilfe von Abbildung \ref{label}
und der folgenden Gleichung:
\begin{align}
g	
\end{align}
die Kapazität $C$ des Kondensators berechnet und dann mit dem vom Kondensator abgelesenen Wert für die Kapazität verglichen.
Der Kondensator hatte nach Herstelller angaben eine Kapazität von \SI{60+-1}{\ohm}.
Dieser Wert liegt in der $1\sigma$-Umgebung des errechneten Wertes: \SI{1+-1}{\ohm} und stimmt somit gut mit dem Theoretischen Wert überein.
Um die UNsicherheiten der Werte zu erhalten wurde einmal für den Herstellerwert die von diesem angegebene Unsicherheit von $10\%$ nach Gleichung
\ref{eq:su3} abgeschätzt und die Unsicherheit des Messwertes ergibt nach Gleichung \ref{eq:uKond}.
