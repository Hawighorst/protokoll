%19

\section{Schlussfolgerung}
%Hauke
Im zweiten Teil des Experimentes sollte zunächst %Hauke
Danach wurde die Induktivität einer Spule und die Kapazität eines Kondensators bestimmt.
Letzterer Wert kann dazu genutzt werden die Ergebnisse zu überprüfen da die Kapazität des Kondensators auch auf diesem angegeben war.
Da sich \SI{60+-3.46}{µF} in der $1\sigma$-Umgebung von \SI{57.4+-3.8}{µF} befindet, ist davon auszugehen das die zuvor errechneten Werte ebenfalls in etwa passen.
Dieses Experiment diente dazu das man anhand einer einfachen Schaltung lernt wie wie man z.B. die Kapazität oder die Induktivität berechnet beziehungsweise was für Werte man durch Messungen bei Gleich- bzw. Wechselstrom überhaupt durchführen kann.










