%19

\section{Schlussfolgerung}


Im ersten Abschnitt wird der Einfluss einer Schaltung auf die Spannungsquelle untersucht, da im Allgemeinen die Klemmspannung der Quelle abhängig von der angeschlossenen Last ist. Dieser Sachverhalt wird im ersten Abschnitt mit Batterien simuliert welche, zur Verdeutlichung mit einem Vorwiderstand versehen sind. Bei dem gegebenen Aufbau beträgt der "Innenwiderstand" $\SI{17.7+-0.6}{\ohm}$ und die Leerlaufspannung $U_E=\SI{1.27+-0.02}{V}$ für einen Akkumulator. Für die Reihenschaltung wurde für Spannung und Innenwiderstand eine Verdreifachung der Werte erwartet. Die Parallelschaltung sollte den Innenwiederstand dritteln und die Leerlaufspannung der einzelnen Monozelle aufweisen. Diese Annahmen wurden durch die Messergebnisse gemäß den Tabellen~\ref{tab:batt-U-R} und \ref{tab:batt-U-P} bestätigt.

%Hauke
Im zweiten Teil des Experimentes soll zunächst die Schaltung zur Leistungsaufnahme mit einem Widerstand getestet werden. Hierbei wird festgestellt, dass das Wattmeter einen geringeren Wert misst, als Volt- und Amperemeter. Bei dem Wechselstromkreis beträgt das Verhältnis \SI{93.5+-0.8}{\percent} und bei dem Gleichstromkreis \SI{ 95.0+-0.8}{\percent}.
Dies deckt sich mit der Beobachtung, dass die Leerlaufleistung des Voltmeters zu einem negativen Ausschlag bei der Leistungsmessung führt. Für weitere Messungen müssten andere Messinstrumente verwendet werden oder die gegenseitige Beeinflussung der Instrumente berücksichtigt werden.
Danach wurde die Induktivität einer Spule und die Kapazität eines Kondensators bestimmt.
Letzterer Wert kann dazu genutzt werden die Ergebnisse zu überprüfen da die Kapazität des Kondensators auch auf diesem angegeben war.
Da sich \SI{60+-3}{\mu F} in der $1\sigma$-Umgebung von \SI{57+-4}{\mu F} befindet, ist davon auszugehen, dass die zuvor errechnete Induktivität ebenfalls im Bereich der theoretischen Werte liegt.
Dieses Experiment diente dazu, dass man anhand einer einfachen Schaltung lernt, wie z.B. die Kapazität oder die Induktivität berechnet werden. Des weiteren werden die Messgrößen und Messverfahren erläutert um Gleich- bzw. Wechselstromkreise zu charakterisieren.










