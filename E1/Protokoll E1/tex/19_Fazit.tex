%19

\section{Schlussfolgerung}


Im ersten Abschnitt wird der Einfluss einer Schaltung auf die Spannungsquelle untersucht, da im Allgemeinen die Klemmspannung der Quelle abhängig von der angeschlossenen Last ist. Dieser Sachverhalt wird im ersten Abschnitt mit Batterien simuliert welche, zur Verdeutlichung mit einem Vorwiderstand versehen sind. Bei dem gegebenen Aufbau beträgt der "Innenwiderstand" $\SI{17.7+-0.6}{\ohm}$ und die Leerlaufspannung $U_E=\SI{1.27+-0.02}{V}$ für einen Akkumulator. Des weiteren werden die theoretischen Vorhersagen für ein Parallel- bzw. Reihenschaltung aus drei Akkumulatoren bestätigt.

%Hauke
Im zweiten Teil des Experimentes sollte zunächst %Hauke
Danach wurde die Induktivität einer Spule und die Kapazität eines Kondensators bestimmt.
Letzterer Wert kann dazu genutzt werden die Ergebnisse zu überprüfen da die Kapazität des Kondensators auch auf diesem angegeben war.
Da sich \SI{60+-3.46}{\mu F} in der $1\sigma$-Umgebung von \SI{57.4+-3.8}{\mu F} befindet, ist davon auszugehen die zuvor errechnete Induktivität ebenfalls im Bereich der theoretischen Werte liegt
Dieses Experiment diente dazu das man anhand einer einfachen Schaltung lernt wie wie man z.B. die Kapazität oder die Induktivität berechnet beziehungsweise was für Werte man durch Messungen bei Gleich- bzw. Wechselstrom überhaupt durchführen kann.










