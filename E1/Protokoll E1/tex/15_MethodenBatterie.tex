



\section{Innenwiderstand einer Batterie}

Es wird im Folgenden der Innenwiderstand einer Schaltung aus Akkumulatoren bestimmt. Zur Verdeutlichung des Effektes ist vor jeden Akkumulator ein zusätslicher Widerstand fest eingebaut. 

\subsection{Methoden}


Zur Bestimmung des Innenwiderstandes wird die Klemmspannung der Spannungsquelle für verschiedene Außenwiderstände gemessen. Aus Spannung und Widerstand wird die Spannung $U$ in Abhängigkeit der Stromstärke $I$ (\cref{fig:batt-ges-u}) und die Leistung $P$ in Abhängigkeit des Außenwiderstandes $R_a$ (\cref{fig:batt-ges-p}) berechnet. Aus den Ausgleichskurven folgen jeweils die Klemmspannung ohne Last $U_0$ sowie der Innenwiderstand $R_i$. Betrachtet werden als Spannungsquelle: eine einzelne Monozelle, eine Parallelschaltung sowie eine Reihenschaltung aus drei Monozellen. 

Aus der Ableseungenauigkeit des Voltmeters folgt als Standardunsicherheit u(U)=\SI{0.2}{V}, die relative Unsicherheit der Steckwiderstände wird mit 5\% abgeschätzt.




