



\section{Innenwiederstand einer Batterie}

Es sollte der Innenwiederstand einer Schaltung aus Akkumulatoren bestimmt werden. Zur Verdeutlichung des Effektes wurde vor jeden Akkumulator ein Widerstand geschaltet. 

\subsection{Methoden}


Zur Bestimmung des Innenwiederstandes wurde die Klemmspannung der Spannungsquelle für verschiedene Außenwiderstände gemessen. Aus Spannung und Widerstand wurden die Spannung $U$ in Abhängigkeit der Stromstärke $I$ (\cref{fig:batt-ges-u}) und die Leistung $P$ in Abhängigkeit des Außenwiederstandes $R_a$ (\cref{fig:batt-ges-p}) berechnet. Aus den Ausgleichskurven folgen jeweils die Klemmspannung ohne Last $U_0$ sowie der Innenwiederstand $R_i$. Betrachtet wurden als Spannungsquelle: eine einzelne Monozelle, eine Parallelschaltung sowie eine Reihenschaltung aus drei Monozellen. 

Aus der Ableseungenauigkeit des Voltmeters folgt als Standardunsicherheit u(U)=\SI{0.2}{V}, die relative Unsicherheit der Steckwiederstände wurde mit 5\% abgeschätzt.




