\subsection{Daten und Analyse}


Aus den Messpunkten $U(R_a)$ folgt mit dem Ohmschen Gesetz \cref{fig:batt-ges-u}. Aus $U_{Kl}=U_0-R_a I$ folgt, dass die Steigung des Ausgleichsgerade dem negativen des Innenwiderstandes entspricht. Ohne Stromfluss gilt $U_0=U_{Kl}$, deswegen entspricht der Y-Achsenabschnitt der Leerlaufspannung $U_0$ der "idealen Spannungsquelle" \cite{lw}. Die aus den Parametern der Anpassungsgerade gefundenen Werte sind in \cref{tab:batt-U-R} dargestellt.





\begin{table}
	\caption{Leerlaufspannung und Innenwiderstand der Spannungsquellen aus den Kennlinien}
	\centering
	\begin{tabular}{|l l||c|c|}
		\hline 
		Schaltung & Index	& Leerlaufspannung $U_0$ & Innenwiederstand $R_i$ \\ 
		\hline \hline
		Einzelne Monozelle	&E& \SI{1.28+-0.01}{V}  & \SI{17.7+-0.4}{\ohm } \\ 
		\hline  
		Parrallelschaltung	&P& \SI{1.289+-0.003}{V } &\SI{5.99+-0.06}{\ohm }  \\ 
		\hline   
		Reihenschaltung	&R& \SI{4.03+-0.12}{V } &\SI{57+-3}{\ohm }  \\ 
		\hline 
	\end{tabular} 
	
	\label{tab:batt-U-R}
	
\end{table}


 


\begin{figure}[h]
	\centering
	\includegraphics[width=0.9\linewidth]{"auswertung/Auswertung Innenwiderstand/Batterie Gesamt U"}
	\caption{Spannungsverläufe der Monozelle~$U_E$, der Parallelschaltung von drei Monozellen~$U_P$ und der Reihenschaltung von drei Monozellen~$U_R$ in Abhängigkeit der Stromstärke~$I$.}
	\label{fig:batt-ges-u}
\end{figure}










\FloatBarrier
%Leistung

\begin{figure}[h]
	\centering
	\includegraphics[width=0.9\linewidth]{"auswertung/Auswertung Innenwiderstand/Batterie Gesamt P"}
	\caption{Leistung $P$ am Lastwiderstand $R_a$ in dessen Abhängigkeit}
	\label{fig:batt-ges-p}
\end{figure}



\begin{table}
	\caption{Leerlaufspannung und Innenwiderstand der Spannungsquellen aus der Leistung}
	\centering
	\begin{tabular}{|l l||c|c|}
		\hline 
		Schaltung & Index	& Leerlaufspannung $U_0$ & Innenwiederstand $R_i$ \\ 
		\hline \hline
		Einzelne Monozelle	&E& \SI{1.27+-0.02}{V}  & \SI{17.6+-0.6}{\ohm } \\ 
		\hline  
		Parrallelschaltung	&P& \SI{1.282+-0.007}{V } &\SI{5.91+-0.09}{\ohm }  \\ 
		\hline   
		Reihenschaltung	&R& \SI{4.26+-0.21}{V } &\SI{63+-5}{\ohm }  \\ 
		\hline 
	\end{tabular} 
	
	\label{tab:batt-U-P}
	
\end{table}


%In analoger Weise zu \cref{fig:batt-ges-u} wurde \cref{fig:batt-ges-p} erstellt. 
Die Leistung am äußeren Widerstand ist gegeben durch 
\begin{align}
 P &=\frac{U_{Kl}^2}{R_a} \label{eq:puk}\\
 &= U_0^2 \frac{R_a}{(R_a+R_i)^2} \label{eq:pu0}.
\end{align} 
Gleichung~\ref{eq:puk} wurde verwendet um die Leistungen zu berechnen, die Ausgleichskurve wurde nach Gleichung~\ref{eq:pu0} erstellt.
Die Werte für $U_0$ und $R_i$ ergeben sich aus der Ausgleichskurve und sind in \cref{tab:batt-U-P} dargestellt. Die Maximale Leistung ergibt sich, bedingt durch den gewählten Ansatz in Gleichung~\ref{eq:pu0}, für $R_a=R_i$. 
