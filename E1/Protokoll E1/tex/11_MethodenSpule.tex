\section{Eigenschaften einer Spule}
In diesem Kapitel werden der Phasenwinkel $\phi$, der Wirkwiderstand R$_\text{W}$ sowie die Induktivität L einer Spule berechnet

\subsection{Methoden}
Um die oben genannten Größen zu berechnen wurde die Spannung U, der Strom I und die Leistung P für den in Abb. \ref{label} zu sehenden Stromkreis gemessen. Es handelt sich hier um die Position b).
Die Spannung und der Strom wurden sowohl bei Wechselstrom als auch bei Gleichstrom bestimmt, während die Leistung nur bei Wechselstrom gemessen wurde. Zu beachten ist dass es sich bei allen im weiteren genannten Werte für U,I , die bei Wechselstrom gemessen wurden, um Effektivwerte handelt und P nur gemittelt angegeben werden kann.
Die Messungen wurden mit einem Multimeter, einem Ampermeter und einem Wattmeter durchgeführt.
All diese Messgeräte wahren mit einem analogen Skala versehen. Aus diesem Grund sind alle Unsicherheiten der Messwerte durch eine Dreiecksverteilung abzuschätzen. 
