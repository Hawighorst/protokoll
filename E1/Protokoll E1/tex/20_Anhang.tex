


\subsection{Verwendete Hilfsmittel}
Die Plots werden mit Python erstellt. Für Anpassungen wird der  Levenberg–Marquardt Algorithmus verwendet. Die Fehler werden nach Empfehlung des "GUM", insbesondere mit Hilfe der gaußschen Fehlerfortpflanzung berechnet. So nicht anders angegeben, werden die theoretischen Zusammenhänge und Gleichungen \cite{lw} entnommen.


\subsection{Unsicherheiten}\label{kap:Unsich}
Kapazität $C$:
\begin{align*}
	u(C)=\frac{1}{\omega} \sqrt{\frac{- R^{2} u(R)^{2} - Z^{2} u(Z)^{2} + \omega^2 u(L)^{2} \left(R^{2} - Z^{2}\right)}{\left(R^{2} - Z^{2}\right) \left(\omega L + \sqrt{- R^{2} + Z^{2}}\right)^{4}}}	.
\end{align*}
Induktivität $L$:
\begin{align*}
	u(L)=\frac{1}{\omega} \sqrt{\frac{- R^{2} u(R)^{2} - Z^{2} u(Z)^{2}}{R^{2} - Z^{2}}}.
\end{align*}
Phasenwinkel $\phi$: 
\begin{align*}
	u(\phi)=\sqrt{\frac{-u\left(\frac{\Delta P}{\Delta UI}\right)^2}{\phi^2-1}}.
\end{align*}
Wirkwiderstand $R_W$:
\begin{align*}
	u(R_W)=\sqrt{\frac{1}{\phi^2-1}\cdot(-Z^2u(\phi)^2+u(Z)^2\cdot(\phi^22-1)\cdot \arccos(\phi)^2))}.
\end{align*}
Werte die sich aus  den Steigungen der Abbildungen entnehmen lassen, sind auch mit der Unsicherheit dieser Anpassung versehen.\\
Rechteckverteilung:
\begin{align}
	u=\frac{a}{2\sqrt{3}}\label{eq:sur}
\end{align}

\subsection{Gleichungen aus der Einführung}
$A_= =\frac{1}{T}\int_{0}^{T}|A(t)|\text{d}t$\\
Für $A(t)=I_0\sin (\omega t)$:
\begin{align*}
I_=  &=\frac{1}{T}\int_{0}^{T}|I_0\sin (\omega t|\text{d}t\\
&=\frac{\omega}{2\pi}\int_{0}^{\frac{2\pi}{\omega}}|I_0\sin (\omega t)| \text{d}t\\
&=\frac{\omega}{\pi}\int_{0}^{\frac{\pi}{\omega}}I_0\sin (\omega t) dt\\
&=\frac{1}{\pi}\left[ I_0\cos(\omega t) \right]_0^{\frac{\pi}{\omega}}\\
&=\frac{2I_0}{\pi}
\end{align*}
$A_{eff}=\sqrt{\frac{1}{T}\int_{0}^{T} A^2(t)dt}$\\
Für $A(t)=I_0\sin (\omega t)$:
\begin{align*}
I_{eff}&=\sqrt{\frac{1}{T}\int_{0}^{T} A^2(t)dt}	\\
&=\sqrt{\frac{\omega}{2\pi}\int_{0}^{\frac{2\pi}{\omega}}I_0^2\sin^2 (\omega t)\text{d}t}\\
&=I_0\sqrt{\frac{\omega}{2\pi}\left[  \frac{-\sin(\omega)\cos(\omega)}{4\omega}+\frac{t}{2} \right]_0^{\frac{2\pi}{\omega}}\text{d}t}\\
&=I_0\sqrt{\frac{1}{2}}
\end{align*}