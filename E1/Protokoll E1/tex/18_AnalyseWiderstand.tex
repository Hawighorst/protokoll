\subsection{Analyse} 

In \cref{fig:widerstand-r2} ist zu erkennen, dass Spannung und Stromstärke proportional zueinander sind. Die Steigung der Ausglichsgeraden gibt den Widerstand an. Bei Gleichstrom betrug der Wiederstand \SI{27.8+-0.7}{\ohm}, bei Wechselstrom \SI{26.1+- 0.5}{\ohm}. Der Wiederstand hat laut Hersteller \SI{27}{\ohm}. Folglich stimmen alle drei Angaben im Rahmen des $2\sigma-$Intervalls überein.


\begin{figure}[h]
	\centering
	\includegraphics[width=0.9\linewidth]{"auswertung/Auswertung Innenwiderstand/Widerstand R2 B"}
	\caption{Relation von Spannung und Stromstärke an dem Verbrauchswiderstand $R_2$.}
	\label{fig:widerstand-r2}
\end{figure}


In \cref{fig:widerstand-r2} ist zu erkennen, dass Spannung und Stromstärke proportional zueinander sind. Die Steigung der Ausglichsgeraden gibt den Widerstand an. Bei Gleichstrom betrug der Wiederstand \SI{27.8+-0.7}{\ohm}, bei Wechselstrom \SI{26.1+- 0.5}{\ohm}. Der Wiederstand hat laut Hersteller \SI{27}{\ohm}. Folglich stimmen alle drei Angaben im Rahmen des $2\sigma-$Intervalls überein.





\begin{figure}
	\centering
	\includegraphics[width=0.9\linewidth]{"auswertung/Auswertung Innenwiderstand/Leistung 2"}
	\caption{}
	\label{fig:leistung-r2}
\end{figure}
