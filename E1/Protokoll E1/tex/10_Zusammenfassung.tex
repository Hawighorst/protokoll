%Zusammenfassung in unter 200 Wörtern

\section{Zusammenfassung}
Im ersten Abschnitt wird der Einfluss einer Schaltung auf die Spannungsquelle untersucht, da im Allgemeinen die Klemmspannung der Quelle abhängig von der angeschlossenen Last ist. Dieser Sachverhalt wird im ersten Abschnitt mit Batterien simuliert welche, zur Verdeutlichung mit einem Vorwiderstand versehen sind. Bei dem gegebenen Aufbau beträgt der "Innenwiderstand" $\SI{17.7+-0.6}{\ohm}$ und die Leerlaufspannung $U_E=\SI{1.27+-0.02}{V}$ für einen Akkumulator. Des weiteren werden die theoretischen Vorhersagen für ein Parallel- bzw. Reihenschaltung aus drei Akkumulatoren bestätigt.

Im zweiten Teil des Protokolls werden verschiedene Schaltungen  $\left[ a),b),c) \text{siehe Abb. \ref{fig:Leistungsaufnahme}} \right]$ behandelt.
%Hauke
In Teil b) geht es hauptsächlich um die Berechnung der Induktivität einer Spule.
Der errechnete Wert liegt bei $L= \SI{0,060+-0,004}{H}$. Leider ist an dieser Stelle noch nicht möglich diesen Wert zu bewerten, da es einen Vergleichswert gibt.
Da jedoch in der aus der Schaltung c) mithilfe der Induktivität die Kapazität eines zusätzlich in Reihe geschalteten Kondensators berechnet wird und dieser Wert mit im Bereich der Unsicherheiten von dem abgelesenen Wert liegt, ist anzunehmen, dass die Induktivität durchaus im Bereich der theoretischen Induktivität liegt.
%\cite{anleitung-ss2015}
%\cite{lw}


