%Zusammenfassung in unter 200 Wörtern

\section{Zusammenfassung}
Im ersten Abschnitt wird der Einfluss einer Schaltung auf die Spannungsquelle untersucht, da im Allgemeinen die Klemmspannung der Quelle abhängig von der angeschlossenen Last ist. Dieser Sachverhalt wird im ersten Abschnitt mit Batterien simuliert welche, zur Verdeutlichung mit einem Vorwiderstand versehen sind. Bei dem gegebenen Aufbau beträgt der "Innenwiderstand" $\SI{17.7+-0.6}{\ohm}$ und die Leerlaufspannung $U_E=\SI{1.27+-0.02}{V}$ für einen Akkumulator. Des weiteren werden die theoretischen Vorhersagen für ein Parallel- bzw. Reihenschaltung aus drei Akkumulatoren bestätigt.

Im zweiten Teil des Protokolls werden verschiedene Schaltungen  $\left[ a),b),c) \text{siehe Abb. \ref{fig:Leistungsaufnahme}} \right]$ behandelt.
Zunächst wird die Leistungsaufnahme an einem Widerstand betrachtet. Die Beziehungen $U=RI$ und $P=UI$ werden bestätigt.
Danach wurde über diese Zusammenhänge die Induktivität einer Spule berechnet.
Der errechnete Wert liegt bei $L= \SI{0,060+-0,004}{H}$. Leider ist es wegen einem fehlendem Vergleichswert nicht möglich dieses Ergebnis zu exakt zu bewerten. 
Da jedoch  mithilfe der Induktivität die Kapazität eines zusätzlich in Reihe geschalteten Kondensators berechnet wird und dieser Wert mit \SI{57.4+-3.8}{\mu F} im Bereich der Unsicherheiten von dem abgelesenen Wert \SI{60+-3.46}{\mu F} liegt, ist anzunehmen, dass die Induktivität im Bereich der theoretischen Induktivität liegt.
%\cite{anleitung-ss2015}
%\cite{lw}



%Im ersten Abschnitt wird der Einfluss einer Schaltung auf die Spannungsquelle untersucht, da im Allgemeinen die Klemmspannung der Quelle abhängig von der angeschlossenen Last ist. Dieser Sachverhalt wird im ersten Abschnitt mit Batterien simuliert welche, zur Verdeutlichung mit einem Vorwiderstand versehen sind. Bei dem gegebenen Aufbau beträgt der "Innenwiderstand" $\SI{17.7+-0.6}{\ohm}$ und die Leerlaufspannung $U_E=\SI{1.27+-0.02}{V}$ für einen Akkumulator. Des weiteren werden die theoretischen Vorhersagen für ein Parallel- bzw. Reihenschaltung aus drei Akkumulatoren bestätigt.\\



%Im zweiten Teil des Protokolls werden verschiedene Schaltungen  $\left[ a),b),c) \right]$ behandelt.
%Zunächst wird die Leistungsaufnahme an einem Widerstand betrachtet. Die Beziehungen $U=RI$ und $P=UI$ werden bestätigt.
%bitte nimm statt b) c) etc. Physikalisch motivierte Beschreibungen!!! Statt R-> Spule, Reihenschaltung aus Spule und Kondesator oder ähnliches(Induktivität. ergänzt durch Kapzität)
%In Teil b) geht es hauptsächlich um die Berechnung der Induktivität einer Spule.
%Der errechnete Wert liegt bei $L= \SI{0,060+-0,004}{H}$. Leider ist an dieser Stelle noch nicht ersichtlich ob dieser Wert 
%logisch ist oder nicht. %logisch ist in dem kontext unpassend
%Da jedoch in der aus der Schaltung c) mithilfe der Induktivität die Kapazität eines zusätzlich in Reihe geschalteten Kondensators berechnet wird und dieser Wert mit im Bereich der Unsicherheiten von dem abgelesenen Wert liegt ist anzunehmen das die Induktivität durchaus im Bereich der theoretischen Induktivität liegt.
%\cite{anleitung-ss2015}
%\cite{lw}
%>>>>>>> b885abb5c84bcfea2cd4620fdfce127513d11e1a
