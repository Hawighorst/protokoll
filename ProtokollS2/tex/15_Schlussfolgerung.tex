\section{Schlussfolgerung}
Bis auf einen Messwert liegen alle Bestwerte in einem Bereich zwischen \SI{9,78}{m/s^2} und \SI{9,89}{m/s^2}. Bei dem abweichenden Messwert ist von einem groben Fehler in der Durchführung auszugehen.\\
Somit bestätigen die vorliegenden Messungen die Werte der PTB und widersprechen mehrheitlich den vorausgegangenen Messungen der Universität Münster. Mögliche Ursache für einen Fehler in der Auswertung der Fallzeit im Fallrohr ist die Annahme, dass die Kugel auf Höhe der ersten Lichtschranke ruhe, bei der Demonstration war dies jedoch offensichtlich nicht der Fall, da die Kugel deutlich oberhalb der Lichtschranke fallengelassen wurde.\\
Abschließend ist anzumerken, dass der Wert des PTB plausibel erscheinen und die Messungen mithilfe des Fallrohres neu ausgewertet werden sollten und ggf weitere Fehlerquellen identifiziert werden müssen.