\subsection{Pendel mit unterschiedlichen Längen}
\begin{figure}
	\begin{tabular}{|c|c|c|c|c|}
		\hline
		Pendel Länge & Anzahl der & Anzahl der  &  $\varnothing$ Gesamtdauer &  $\varnothing $Dauer einer  \\
		in m &  Messungen & Schwingungen & in s & Schwingung in s \\
		\hline
		\hline
		0,774 & 3 & 10 & 17,64 & \SI{1,76+-1,057E-5}\\
		\hline
		0,844 & 3 & 10 & 18,44 & \SI{1,84+-9,434E-6} \\
		\hline
		0,995 & 3 & 10 & 19,72 & \SI{1,97+-7,642E-6} \\
		\hline
		1,012 & 3 & 10 & 20,10 & \SI{2,01+-1,510E-6} \\
		\hline
		1,058 & 3 & 10 & 20,66 & \SI{2,07+-2,547E-6 \\
		\hline
	\end{tabular}
\caption{Diese Tabelle 	zeigt die Dauer der Pendelschwingungen zu den unterschiedlichen Pendellängen an. Diese Pendellängen sind mit einer Unsicherheit von $\SI{+- 0,0012}{m}$
	 angegeben. }
\label{lversch.}
\end{figure}
Diese Messreihe diente genauso wie die erste Messreihe dazu die Gravitationsbeschleunigung rechnerisch aus der Schwingungsdauer eines Pendels zu bestimmen.
Hierzu wurden mehrere Messung der Schwingungsdauer bei Variierender Länge des Pendels durchgeführt.
Damit man die Messdaten auswerten kann muss man zunächst die möglichen Fehlerquellen ausfindig machen und die Unsicherheiten berechnen die von ihnen ausgehen.
Diese Unsicherheiten kann man zunächst in drei Teilbereiche aufteilen: 
\begin{itemize}
	\item Unsicherheiten die bei der Berechnung der Schwingungsdauer auftreten.
	Diese Unsicherheiten werden mit einer Typ A Unsicherheitsberechnung nach Gleichung \ref{sigma} mit $t_p = 1,32$ abgeschätzt und sind in Tabelle \ref{lversch.} angegeben.
	\item Unsicherheiten beim messen der zeit einer Pendelschwingung\\
	Die Reaktionszeit beim stoppen der zeit und die digitale anzeige der Stoppuhr sind die Unsicherheiten die bei der Zeitmessung auftreten können. Bei der Reaktionszeit handelt es sich um eine Typ B Unsicherheit die nach Formel \ref{su3} mit $a=0,2$ abgeschätzt wird, da hier angenommen wird das die Reaktionszeit bei\SI{+- 0,1}{s} liegt.Die jeweiligen werte der Unsicherheit in Tabelle \ref{Tab:Uversch} eingetragen. Da die Stoppuhr immer auf Millisekunden rundet wird die Unsicherheit beim Ablesen der Zeit  mit einer Rechteckverteilung nach Gleichung \ref{sur} (mit $a=\SI{0,01}{s}$) und $N=10$abgeschätzt,  somit kommt man auf eine Unsicherheit von \SI{+-0,0028}{s}.

	\item Unsicherheiten beim messen der Länge der verschiedenen Pendel
	Bei dem Messen der Länge kann ein ablese Fehler aufgrund des Analogen Maßbandes entstehen. Dementsprechend wird diese Unsicherheit nach Gleichung \ref{su3} mit $a= \SI{0,06}{m}$ abgeschätzt und man kommt auf die Unsicherheit bei der Pendellänge die in Tabelle \ref{Tab:Uversch} zu sehen ist. 

\end{itemize}
 Die kombinierte Standardunsicherheit wird nach Gleichung\ref{kombsu} mit den Werten  aus der Tabelle \ref{Tab:Uversch} berechnet.
 Hierbei handelt es sich um die Endgültige Unsicherheit mit der dann auch der im nachfolgendem berechnete Wert für die Gravitationsbeschleunigung betrachtet werden muss. 
 
Dieser letzte und entscheidende Wert, der ebenfalls in Tabelle \ref{Tab:Uversch} zusehen ist, ist die Gravitationsbeschleunigung die nach Gleichung \ref{bestg} bestimmt wird.
Diese werte sind in \ref{Bild:VPL} aufgetragen. Die y-Achse auf der die Schwingungsdauer aufgetragen ist wurde quadriert um einen Linearen Zusammenhang zu erzeugen. Um diesen Zusammenhang zu verdeutlichen wurde eine Lineare Anpassung (Nach der Methode der kleinsten Quadrate) mit Steigung m=\SI{3,99 +- 0,26}{s^2/m} verwendet. Betrachtet man nun diese Abbildung so sieht man das zwei Messwerte stark von dem Idealwert abweichen. Dies ist vermutlich auf Messfehler bei der Durchführung des Experimentes zurückzuführen. Vor allem der Wert für die Pendellänge von \SI{0,995}{m} mit g=\SI{10,1}{m/s^2}, weicht sehr stark vom erwarteten wert für g=\SI{9,813}{m/s^2} ab. Dies führt dann dazu das der Wert für die Gravitationsbeschleunigung für die Pendellänge von \SI{1,058}{m} ebenfalls stark vom Idealwert abweicht.
\begin{figure}
	\begin{tabular}{|c|c|c|c|c|c|}
		\hline
		Länge des & Unsicherheit der   & Unsicherheit der  & Unsicherheit   & Kombinierte  & g in \\
		 Pendels & Digitalen Anzeige& Analogen Anzeige &durch die &Unsicherheit & \SI{}{m/s^2} \\
		in m &   der Stoppuhr in s &   des Maßbandes in m &  Reaktionszeit in s & in \SI{}{m/s^2} & \\
		\hline
		0,774 &  \SI{+-0,002} & \SI{+-0,0012} & \SI{+-0,0003} & 0,021 & 9,824 \\
		\hline 
		0,844 & \SI{+-0,002} & \SI{+-0,0012} & \SI{+-0,0003}& 0,02 & 9,803 \\
		\hline
		0,995 & \SI{+-0,002} & \SI{+-0,0012} & \SI{+-0,0003}  &0,019 & 10,10  \\
		\hline 
		1,012 & \SI{+-0,002} & \SI{+-0,0012} & \SI{+-0,0003}  &0,019 & 9,886 \\
		\hline
		1,058 &  \SI{+-0,002} & \SI{+-0,0012} & \SI{+-0,003} & 0,023 & 9,786 \\
		\hline
		\end{tabular}
	    \caption{Diese Tabelle führt alle Ungenauigkeiten der einzelnen Messungen, der Kombinierten Unsicherheit sowie des Errechneten Wertes von "g" auf.}
	   \label{Tab:Uversch}
\end{figure}
In Abb. \ref{Bild:VPL} wird die Dauer einer Schwingung zum Quadrat in Abhängigkeit von der Pendellänge dargestellt. Anhand der Anpassung die mithilfe der Funktion: $y=mx+b$ mit $m= \SI{3,99+-0,26}{s^2/m}$ und b=$\SI{0,0039+-0,2758}{s^2}$ sieht man das genau zwei werte von der Anpassungsfunktion abweichen. Nämlich die Pendellängen 0,955 m und 1,058 m wobei der letztere Wert nur deshalb so stark abweicht da die Pendellänge 0,955 m mit in die Lineare Anpassung einbezogen wurde.   
\begin{figure}
	\includegraphics[width=0.8\textwidth]{Pendel/PlotLängenV.pdf}
	\caption{In der Grafik ist die Dauer einer Schwingung zum Quadrat in Abhängigkeit der Pendellänge, sowie eine Lineare Anpassung  zu sehen.}
	\label{Bild:VPL}
\end{figure}
 