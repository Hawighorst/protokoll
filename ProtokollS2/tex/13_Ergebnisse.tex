\newpage

\section{Ergebnisse und Diskussion}

%Ergebnisse
%Analyse
%interpretation und Einordnung eigener\setion ?

\subsection{Pendel mit konstanter Länge}


\begin{figure}[h]


\begin{tabular}{|c||c|c|c|}
\hline
Messung Nr. & Anzahl der Schwingungen & Gesamtdauer [s] & Dauer einer Schwingung $T$ [s] \\ \hline \hline
1&	20&	42,94&	2,147	\\ \hline
2&	20&	42,94&	2,147	\\ \hline
3&	20&	42,91&	2,1455	\\ \hline
4&	20&	42,91&	2,1455	\\	\hline
5&	20&	42,97&	2,1485	\\ \hline
Durchschnitt&&42,934&2,1467	\\ \hline


	
\end{tabular}
\caption{Die Tabelle gibt die Anzahl der Pendelschwingungen sowie die zugehörige Dauer an. Die Länge des Pendels betrug \SI{1,145+-0,0012} {\m}}
\label{lkonst}

\end{figure}


Berechnung Unsicherheit Länge \\
Berchnung Unsicherheit Zeit Typ A, B



