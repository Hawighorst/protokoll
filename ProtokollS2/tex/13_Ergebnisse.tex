\newpage

\section{Ergebnisse und Diskussion}

%Ergebnisse
%Analyse
%interpretation und Einordnung eigener\setion ?

\subsection{Pendel mit konstanter Länge}


\begin{figure}[h]


\begin{tabular}{|c||c|c|c|}
\hline
Messung Nr. & Anzahl der Schwingungen & Gesamtdauer [s] & Dauer einer Schwingung $T$ [s] \\ \hline \hline
1&	20&	42,94&	2,147	\\ \hline
2&	20&	42,94&	2,147	\\ \hline
3&	20&	42,91&	2,1455	\\ \hline
4&	20&	42,91&	2,1455	\\	\hline
5&	20&	42,97&	2,1485	\\ \hline
Durchschnitt&&42,934&2,1467	\\ \hline


	
\end{tabular}
\caption{Die Tabelle gibt die Anzahl der Pendelschwingungen sowie die zugehörige Dauer an. Die Länge des Pendels betrug \SI{1,145+-0,0012} {\m}}
\label{lkonst}

\end{figure}

Zur Bestimmung der Unsicherheiten wurde angenommen, dass die Pendellänge um nicht mehr als \SI{3}{\mm} von dem gemessenen Wert abweicht, hierdurch ergibt sich mit Approximation der Wahrscheinlichkeitsdichtefunktion durch eine Dreiecksverteilung nach Gleichung \ref{su3} mit $a=\SI{6}{\mm}$ eine Standartunsicherheit von \SI{1,2}{mm}. \\
Bei der Zeitmessung wurden zwei Unsicherheitsquellen identifiziert, zum einen die Reaktionszeit von etwa 0,1 Sekunden je Messung, sowie die Standartabweichung aufgrund der Schwankungen der gemessenen Zeiten. Die Unsicherheit des Typs B, für  $a=\SI{0,1}{s} $ und $N=20$ wird nach Gleichung \ref{sur} mit \SI{1,4e-3}{\s} abgeschätzt. Die Unsicherheit des Typs A ergibt sich aus Tabelle \ref{lkonst}, $t_p=1,14$ sowie Gleichung \ref{sigma}, sie beträgt \SI{3e-7}{s}. Aus Gleichung \ref{kombsu} folgt mit einsetzten der obigen Werte das die Unsicherheit des Typs A keinen Einfluss  auf die signifikanten Stellen der kombinierten Unsicherheit bezüglich der Zeit hat, sie beträgt daher ebenfalls \SI{1,4e-3}{mm}.
Mit den Gleichungen theorie erhält man: $g=\SI{9,809+-0,015}{m/s^2}$.





