

\section{Ergebnisse und Diskussion}

%Ergebnisse
%Analyse
%interpretation und Einordnung eigener\setion ?

\subsection{Pendel mit konstanter Länge}


\begin{table}[h]

\caption{Anzahl der Pendelschwingungen sowie die zugehörige Dauer an. Die Länge des Pendels betrug \SI{1,145+-0,0012} {\m}}

\begin{tabular}{|c||c|c|c|}
\hline
Messung Nr. & Anzahl der Schwingungen & Gesamtdauer [s] & Dauer einer Schwingung $T$ [s] \\ \hline \hline
1&	20&	42,94&	2,1470	\\ \hline
2&	20&	42,94&	2,1470	\\ \hline
3&	20&	42,91&	2,1455	\\ \hline
4&	20&	42,91&	2,1455	\\	\hline
5&	20&	42,97&	2,1485	\\ \hline
Durchschnitt&&42,93&2,1467	\\ \hline


	
\end{tabular}
\label{lkonst}

\end{table}

Zur Bestimmung der Unsicherheiten wurde angenommen, dass die Pendellänge um nicht mehr als \SI{3}{\mm} von dem gemessenen Wert abweicht, hierdurch ergibt sich mit Approximation der Wahrscheinlichkeitsdichtefunktion durch eine Dreiecksverteilung nach Gleichung \ref{su3} mit $a=\SI{6}{\mm}$ eine Standardunsicherheit von \SI{1,2}{mm}. \\


Bei der Zeitmessung wurden zwei relevante Unsicherheitsquellen identifiziert, zum einen die Reaktionszeit von etwa 0,1 Sekunden je Messung, sowie die Standardabweichung aufgrund der Schwankungen der gemessenen Zeiten. Unsicherheiten aufgrund der Digitalen Anzeige oder Kalibrierung der Stoppuhr sind so klein gegenüber der Ungenauigkeit aufgrund der Reaktionszeit, dass sie zu vernachlässigen sind.
 Die Unsicherheit des Typs B, für  $a=\SI{0,1}{s} $ und $N=20$ wird nach Gleichung \ref{su3} mit \SI{1,0e-3}{\s} abgeschätzt. Die Unsicherheit des Typs A ergibt sich aus Tabelle \ref{lkonst}, mit $t_p=1,14$ sowie den Gleichung \ref{sigma}, sie beträgt \SI{3e-7}{s}. Aus Gleichung \ref{kombsu} folgt mit einsetzten der obigen Werte das die Unsicherheit des Typs A keinen Einfluss  auf die signifikanten Stellen der kombinierten Unsicherheit bezüglich der Zeit hat, sie beträgt daher ebenfalls \SI{1,0e-3}{mm}.
Mit den Gleichungen \ref{bestg} und \ref{kombu} erhält man: $g=\SI{9,809+-0,013}{m/s^2}$.





