\section{Einführung}
Anlass dieses Experimentes, waren Messungen der Universität Münster welche die lokalen Fallbeschleunigung $g$, nach wiederholten Messungen, auf \SI{10,75+-0,25}{m/s^2} beziffern. Dies widerspricht den Angaben der Physikalisch-Technischen Bundesanstalt Braunschweig welche die Fallbeschleunigung für Münster mit $g=\SI{9,813}{m/s^2}$ angibt. Um diese Unterschiede besser beurteilen zu können, sollte die Fallbeschleunigung mit Hilfe eines weiteren Experimentes bestimmt werden. Wie in \cref{theorie} erläutert, eignet sich hierfür das Fadenpendel, da die Periodendauer nur von der Fallbeschleunigung $g$ und dem Abstand des Schwerpunktes von der Aufhängung $l$ abhängen.\\
Zunächst wurden fünfmal je 20 Schwingungsdauern bei einer Länge gemessen. Anschließend wurden 30 Pendelschwingungen bei verschiedenen Längen vermessen. Die zunächst durchgeführten Messungen mit einer Pendellänge ergaben $g=\SI{9,809+-0,013}{m/s^2}$ , die Messungen über verschiedene Längen sind mit dem gegebenen Wert konsistent. Beide Messungen bestätigen die Messungen des PTB.