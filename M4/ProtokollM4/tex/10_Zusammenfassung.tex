%Zusammenfassung in unter 200 Wörtern

\section{Zusammenfassung}
In den Experimenten ging es um die Beschreibung verschiedener, näherungsweise elastischer  Stoßprozesse.  In dem ersten Teil wurden zwei in einer Ebene hängenden Kugeln betrachtet, die nach Auslenkung einer Kugel gegeneinander stoßen. Mithilfe der gemessenen Auslenkung der Kugeln, wurde das Gewichtsverhältnis bestimmt. Hier wurde ein proportionaler Zusammenhang beobachtet. Anschließend wurde eine Kugel auf einer Rinne an Stelle der Auslenkung des einen Pendels verwendet. Hier war die Auslenkung der gestoßenen Kugel proportional zur Wurzel der Höhe.\\
Während des ersten Experiments als Bestätigung der Theorie zu werten ist, liefert das zweite Experiment den theoretischen Wert mit einer Abweichung von etwa $10\%$, diese wird in der Schlussfolgerung diskutiert.


