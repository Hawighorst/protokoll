%19

\section{Schlussfolgerung}
Im Laufe des Experimentes stellte sich heraus, dass sich die Pendelstöße näherungsweise durch elastische Stöße beschreiben lassen (vgl. \cref{kap:Bal}). Versucht man jedoch die Stöße der rollenden Kugel damit zu beschreiben, so stellt man fest das die Ergebnisse deutlich von den erwarteten Werten abweichen. (Vgl. \cref{kap:roll}) .
In dem soeben genannten Beispiel sind die Abweichungen darauf zurückzuführen, dass bei einem elastischen Stoß davon ausgegangen wird das der Impuls vollkommen auf die Zweite Kugel ( die Kugel gegen die die erste Kugel stößt) übertragen wird. Während der Durchführung wurde jedoch beobachtetet das sich beide Kugeln nach dem Stoß bewegten. Dies wurde bei den Berechnungen nicht berücksichtigt.
Der Grund warum dieses Experiment durchgeführt wurde, ist das viele Modelle in der Physik auf sie zurückgreifen um verschiedene Prozesse zu beschreiben. Zum Beispiel den Zusammenprall von Teilchen auf Atomarer Ebene. Bei diesen Prozessen handelt es sich dann zwar nicht um elastische Stöße, das Prinzip ist jedoch ähnlich. Der Fall eines Elastischen Stoßes wurde deshalb gewählt weil er am einfachsten zu beschreiben ist.



% so weit außerhalb des Unsicherheitsintervalls, dass dies nicht statistisch zu erklären ist.
%Es ist davon auszugehen, dass die Näherung als elastischer Stoß für diese Abweichungen verantwortlich ist, da im Experiment zu beobachten war, das die rollende Kugel nach dem Stoß nicht ruhte. Dies müsste entweder in Form größerer Unsicherheiten oder durch Geschwindigkeitsmessung der rollenden Kugel nach dem Stoß berücksichtigt werden. Trotztdem bestätigt











