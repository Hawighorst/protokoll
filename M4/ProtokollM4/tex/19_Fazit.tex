%19

\section{Schlussfolgerung}
Im Experiment stellte sich heraus, dass sich die Pendelstöße näherungsweise durch elastische Stöße beschreiben lassen (vgl. \cref{kap:Bal}). Versucht man jedoch die Stöße der rollenden Kugel damit zu beschreiben, so stellt man fest, dass die Ergebnisse deutlich von den erwarteten Werten abweichen. (Vgl. \cref{kap:roll}) .
In dem soeben genannten Beispiel sind die Abweichungen darauf zurückzuführen, dass bei einem elastischen Stoß davon ausgegangen wird das der Impuls vollständig auf die Kugel am Pendel übertragen wird. Während der Durchführung wurde jedoch beobachtetet, dass sich beide Kugeln nach dem Stoß bewegten. Dies wurde bei den Berechnungen nicht berücksichtigt, da hierzu die Geschwindigkeit der rollenden Kugel gemessen werden müsste und keine passenden Apperaturen hierzu vorhanden waren. Der Fehler durch die Näherung ist daher vermutlich für die Abweichung von ca. $10\%$ verantwortlich.
Das Experiment bot eine Einführung in grundlegende Stoßprozesse.
Stoßprozesse sind die Basis vieler aktueller Experiment so können z.B. die Teilchenbeschleuniger in der Kernphysik Auskunft geben über den Aufbau von Teilchen auf atomaren Skalen. Hier handelt es sich jedoch um Stoßprozesse, die im Allgemeinen nicht elastiosch sind.
Der Fall eines zentralen, elastischen Stoßes wurde deshalb betrachtet, weil er weniger Messungen erfordert und eine gute erste Näherung der beobachteten Stöße darstellt. 





