\section{Methoden}
\subsection*{Balistisches Pendel}
In diesem Teil des Experimentes wurde das Verhalten zweier Unterschiedlich großer Kugeln beim Zusammenprall beobachtet.
Zu diesem Zweck wurden diese Kugeln so aufgehängt, dass ihr Schwerpunkt genau auf einer Ebene lag.
Auf diese weise konnte man den Stoßprozess durch einen idealen Elastischen Stoß nähern.
Es wurden zwei Messreihen aufgenommen: Einmal wurde die Kleine Kugel Ausgelenkt und einmal wurde die Große Kugel ausgelenkt.
Für jede Kugeln wurden fünf verschiedene Auslenkungen beobachtet und pro Auslenkung wurden fünf Messwerte aufgenommen.
Um später beurteilen zu können wie gut sich die Prozesse durch einen elastischen Stoß nähern lassen wurden die Kugeln gewogen.

