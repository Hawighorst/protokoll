\section{Methoden}

Bei beiden Experimenten wurde jeweils einer der Stoßpartner ausgelenkt, während der andere in Ruhe war. Bei dem Ballistischen Pendel war dies ein Pendel, bei der rollenden Kugel eine Metallkugel welche eine Rinne herunter rollt an deren Mündung die Masse des Pendel war.

\subsection*{Balistisches Pendel}
In diesem Teil des Experimentes wurde das Verhalten zweier Unterschiedlich großer Kugeln beim Zusammenprall beobachtet.
Zu diesem Zweck wurden diese Kugeln so aufgehängt, dass ihr Schwerpunkt genau in einer Ebene lag.
Auf diese Weise konnte man den Stoßprozess durch einen zentrale, elastischen Stoß nähern.
Es wurden zwei Messreihen aufgenommen: Zunächst wurde die kleine Kugel ausgelenkt und anschließend wurde die große Kugel ausgelenkt. Die Auslenkung vom Ruhepunkt wurde mithilfe eines Messschiebers gemessen.
Für jede Kugeln wurden fünf verschiedene Auslenkungen betrachtet und pro Auslenkung wurden je fünf Messwerte aufgenommen.
Um beurteilen zu können wie gut sich die Prozesse durch einen elastischen Stoß nähern lassen wurden die Kugeln gewogen.

