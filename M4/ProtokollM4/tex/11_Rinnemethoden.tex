\subsubsection*{Rollende Kugel}

\begin{figure}[h]
	\centering
	\includegraphics[width=0.7\linewidth]{res/FallrinneSkizze}
	\caption{Abbildung der Fallrinne mit den für die Auswertung relevanten Abmessungen\cite{anleitung-ws2017}.}
	\label{fig:rinneskizze}
\end{figure}

<<<<<<< HEAD
Die Kugel wurde jeweils fünfmal aus verschiedenen Höhen auf einer Fallrinne,gemäß Abbildung \ref{fig:rinneskizze}, positioniert und gegen das Pendel am Ende der Fallrinne rollen gelassen. Verwendet wurde das schwerere Pendel der bei dem Versuch zu den ballistischen Pendeln genutzten. Gemessen wurde jeweils die Auslenkung des Pendels$a$ in Abhängigkeit von $S$. Als Ablesehilfe wurde ein Reiter auf einem Messschieber genutzt welcher vor Beginn der Messung jeweils so justiert wurde, dass er etwas hinter der erwarteten Auslenkung $a$ war.
=======
Die Kugel wurde jeweils fünfmal aus verschiedenen Höhen auf einer Fallrinne, gemäß Abbildung \ref{fig:rinneskizze}, positioniert und gegen das Pendel am Ende der Fallrinne rollen gelassen. Gemessen wurde jeweils die Auslenkung des Pendels $a$ in Abhängigkeit von $S$. Als Ablesehilfe wurde ein Reiter auf einem Messschieber genutzt welcher vor Beginn der Messung jeweils so justiert wurde, dass er etwas hinter der erwarteten Auslenkung $a$ war.
>>>>>>> dd24ff0602127c75ad5f99b8026b3324213a4bcf
Die Auslenkung wurde anschließend gemittelt und gegen die Wurzel der Höhe aufgetragen, da die folgenden Zusammenhänge für die Näherung als vollkommen elastischen, zentralen Stoß erwartet wurden:
\begin{align}
a=\frac{2m_1}{m_1+m_2}\sqrt{\varepsilon 2 l h} \label{eq:alenkrinne}
\end{align}
Aus energetischen Betrachtungen und dem Trägheitsmoment der Kugel um die Rotationsachse folgt $\varepsilon= 5/9$, da nur die kinetische Energie  bei dem Stoßprozess übertragen wird und die Rotationsenergie keinen Beitrag liefert.
Des weiteren gilt nach Abbildung \ref{fig:rinneskizze}:
\begin{align}
\sin \alpha &=\frac{H}{H^2+L^2}=\frac{h_0-h}{S} \\
\Rightarrow  h(S) &= h_0-\frac{S}{1+\frac{L^2}{H^2}}
\end{align}













