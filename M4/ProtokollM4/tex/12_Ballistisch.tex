\section{Ballistisches Pendel}\label{kap:Bal}
%\subsection*{Beobachtung und Analyse}
Beobachtet wurde, dass die große Kugel beim Aufprall der kleinen Kugel ausgelenkt wurde und umgekehrt. Außerdem bewegte sich nahezu nur die zweite Kugel nach dem Zusammenstoß weiter. Die Kugel die ausgelenkt wurde, stand nach dem Zusammenprall still. Dies ließ darauf schließen, dass es sich bei dem Stoßprozess um einen nahezu vollkommen elastischen Stoß handelte, welcher im folgenden als solcher behandelt wird.

\begin{figure}[h]
	\centering
	\includegraphics[width=0.8\textwidth]{res/GrosKlein.pdf}
	\caption{Auslenkung der kleinen Kugel in Abhängigkeit der Auslenkung der großen Kugel.}
	\label{fig:grosklein}
\end{figure}
\begin{figure}[h]
	\centering
	\includegraphics[width=0.8\textwidth]{res/KleinGros.pdf}
	\caption{Zu sehen ist die Auslenkung der großen Kugel in Abhängigkeit von der Auslenkung der kleinen Kugel.}
	\label{fig:kleingros}
\end{figure}

Die bei den fünf Messreihen erhaltenen Messwerte wurden jeweils gemittelt und sind in den Abbildungen \ref{fig:grosklein} und \ref{fig:kleingros} zu sehen. Die Fehlerbalken ergeben sich aus den Gleichungen \ref{eq:sud}, \ref{eq:sunv} und \ref{eq:kombsu} wobei von einer Messungenauigkeit von \SI{+-3}{mm} ausgegangen wurde.
Die theoretische Auslenkung nach dem Stoß ist durch
\begin{align}
	a_2= \frac{2m_1}{m_1+m_2} a_1\label{eq:Auslenkung}
\end{align} 

gegeben, wobei $m_1$ die Masse der ausgelenkten Kugel ist und $a_1$ die Auslenkung von $m_1$.\\
Man erkennt in den Abbildungen, dass die Messwerte proportional ansteigen.
Da dies ebenfalls den theoretischen Voraussagen entspricht, wurden die Messwerte mit der proportionalen Anpassung $a_2=b \cdot a_1$ angepasst und die Unsicherheit der Anpassung wurde aus Gnuplot übernommen.
Um die Ergebnisse zu überprüfen wurde die Steigung $b$ aus den Massen der Kugeln bestimmt. Die Unsicherheit der Masse, ist durch die Anzeigeungenauigkeit der digitalen Waage, die auf \SI{0.01}{g} genau anzeigt, gegeben. Aus den Gleichungen \ref{eq:sur} und \ref{eq:kombsu} folgt die Unsicherheit für die aus den Massen berechnete Steigung. Die Werte sind in Tabelle \ref{tab:steigung} aufgeführt.
Man erkennt, dass die Werte zwar voneinander Abweichen, jedoch nur um ca. $3\%$ (Groß gegen Klein) bzw. um ca. $2\%$ (Klein gegen Groß). Dies ist darauf zurückzuführen, dass es sich bei dem untersuchten Stoß nicht um einen perfekten elastischen Stoß handelte oder das die Pendel nicht immer in absoluter Ruhe war.
Über das Verhältnis der Steigungen der Anpassungsfunktion erhält man ein auch das Gewichtsverhältnis, dass sich über: 
\begin{align}
\frac{b_1}{b_2}=\frac{m_1}{m_2}	
\end{align} 
($b_1$, $m_1$ Steigung bzw. ausgelenktes Gewicht aus Abb. \ref{fig:grosklein} und $b_2$,$m_3$ Steigung bzw ausgelenktes Gewicht aus Abb. \ref{fig:kleingros} ) berechnen lässt. die Ergebnisse sind in Tabelle \ref{tab:Gewicht} zu sehen.
Man sieht ,dass das aus den Massen berechnete Gewichtsverhältnis  noch innerhalb der Unsicherheit des aus den Steigungen berechneten Verhältnisses liegt. Dies lässt darauf schließen, dass dieses Verfahren recht genau ist.

\begin{table}[h]
	\caption{Zu sehen sind die Steigungen der in Abb. \ref{fig:grosklein} und \ref{fig:kleingros} zu sehenden Anpassungsgeraden bzw. die Steigung die sie Theoretisch haben sollte.}
	\begin{tabular}{|c|c|c|}
		\hline
		& Steigung aus den & Steigung Experimentell\\
		& Gewichten bestimmt & bestimmt\\
		\hline
		Groß gegen Klein &  \SI{1,487+-0,0002}{} & \SI{1,453+-0,016}{} \\
		\hline
		Klein gegen Groß & $0,513 \pm 6 \cdot 10^{-5}$&\SI{0,502+-0,006}{}\\
		\hline
	\end{tabular}
\label{tab:steigung}
\end{table}

\begin{table}[h]
	\caption{Zu sehen ist hier das Gewichtsverhältnis der Kleinen Kugel zur Großen Kugel. }
	\begin{tabular}{|c|c|}
		\hline
		Gewichtsverhältnis aus  & Gewichtsverhältnis aus der\\
		 den Gewichten &  Experimentell ermittelten Steigung\\
		\hline
		$0,3451 \pm 6 \cdot 10^{-6}$& \SI{0,3455+-0,004}{}\\
		\hline
	\end{tabular}
	\label{tab:Gewicht}
\end{table}
