%19

\section{Schlussfolgerung}
<<<<<<< HEAD
Im ersten Teil der Auswertung wurde der Elastizitätsmodul $E$ bestimmt. Beim Vergleich von den Experimentell bestimmten Werten mit Literaturwerten fiel auf, dass die Vermutung, das es sich bei der Messingfarbenden runden Stange um Messing handelte vermutlich falsch ist, da die Werte für den Elastizitätsmodul 























=======
Mithilfe der Durchbiegung in Abhängigkeit vom Gewicht wurde der Elastizitätsmodul von den vier verschiedenen Stangen bestimmt.
Beim Vergleich von den Experimentell bestimmten Werten mit Literaturwerten fiel auf, dass die Vermutung das es sich bei der Messingfarbenden Runden Stange um Messing handelte vermutlich falsch ist, da der Wert für den Elastizitätsmodul fast identisch mit dem von Kupfer ist. 
Es gibt bei allen Werten Abweichungen dies liegt aber auch daran das es auch Abweichungen bei den Literaturwerten gibt. Des weiteren hängt der Elastizitätsmodul auch von der Art und weise der Verarbeitung der Materialien ab und im falle von Messing und Stahl auch von der Legierung.
Die Experimentell bestimmten Werte sind in der Tabelle \ref{tab:Ela} und die Literaturwerte in der Tabelle \ref{tab:ElaLit} zu finden.
>>>>>>> 3a55c88da677aba14322d7ba36846b810e90b802

Das Torsionspendel eignete sich um den Schubmodul( $G =\SI{7.87+-0.09e10}{kg \per \second \squared  \metre}$) des verwendeten Drahtes bei bekanntem Trägheitsmoment zu bestimmen und durch diese Materialeigenschaft Rückschlüsse auf das verwendete Material (Stahllegierung) zu ziehen. Da der Schubmodul jedoch von dem Fertigungsprozess und den Beimengungen abhängt, ist dies nur als Richtwert zu verstehen. Bei bekanntem Schubmodul $G$ und den Abmessungen des Drahtes bzw. dem Direktionsmodul $D^*=\SI{8.2+-0.2e-5}{\kilogram \metre \squared \per \second \squared }$  lässt sich aus der Schwingungsdauer das jeweilige Trägheitsmoment berechnen.  Die gemessenen  Trägheitsmoment  sind \cref{tab:vglJ} zu entnehmen.








