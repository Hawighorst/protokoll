%19

\section{Schlussfolgerung}
Im ersten Teil der Auswertung wurde der Eleatizitätsmodul $E$ bestimmt. Beim Vergleich von den Experimentell bestimmten Werten mit Literaturwerten fiel auf das die Vermutung das es sich bei der Messingfarbenden Runden Stange um Messing handelte vermutlich falsch ist, da die Werte für den Elastizitätsmodul 
























Das Torsionspendel eignete sich um den Schubmodul( $G =\SI{7.87+-0.09e10}{kg \per \second \squared  \metre}$) des verwendeten Drahtes bei bekanntem Trägheitsmoment zu bestimmen und durch diese Materialeigenschaft Rückschlüsse auf das verwendete Material (Stahllegierung) zu ziehen. Da der Schubmodul jedoch von dem Fertigungsprozess und den Beimengungen abhängt, ist dies nur als Richtwert zu verstehen. Bei bekanntem Schubmodul $G$ und den Abmessungen des Drahtes bzw. dem Direktionsmodul $D^*=\SI{8.2+-0.2e-5}{\kilogram \metre \squared \per \second \squared }$  lässt sich aus der Schwingungsdauer das jeweilige Trägheitsmoment berechnen.  Die gemessenen  Trägheitsmoment  sind \cref{tab:vglJ} zu entnehmen.








