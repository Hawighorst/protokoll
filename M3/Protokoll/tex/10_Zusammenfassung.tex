%Zusammenfassung in unter 200 Wörtern

\section{Zusammenfassung}
Im ersten Teil des Experimentes wurde die Durchbiegung von verschiedenen Stäben in Abhängigkeit von dem angehängten Gewicht bestimmt. Mithilfe dieses Zusammenhanges kann man das Elastizitätsmodul $E$ bestimmen. Da es sich hierbei um eine Materialkonstante handelt dient der Elastizitätsmodul $E$ nicht nur dazu die Elastizität eines Stoffes wiederzugeben sondern er gibt auch die Möglichkeit Rückschlüsse über das vorliegende Material zu ziehen.
Auf diese Weise wurde herausgefunden das es sich bei den untersuchten Stoffen um Kupfer, Aluminium, Stahl, Messing und Aluminium handelte.

In dem zweiten Teil des Experimentes wurde das Torsionspendel untersucht. Bei bekanntem Trägheitsmoment der Masse eignet es sich dazu das Schubmodul $G$ des Drahtes zu bestimmen und Informationen über das verwendete Material zu erhalten. So lies der gemessen Wert von $G=\SI{7.87+-0.09e10}{kg \per \second \squared  \metre}$ eine Stahllegierung vermuten.Bei bekanntem Aufbau und Schubmodul lässt sich alternativ das Trägheitsmoment eines Körpers um die, durch den Draht vorgegebene, Rotationsachse bestimmen. Durch unterschiedliche Aufhängungen der Hantelscheiben wurde der Einfluss des Abstandes zwischen Schwerpunkt und Rotationsachse untersucht und mithilfe des Steinerschen Satzes ausgewertet.