%Zusammenfassung in unter 200 Wörtern

\section{Zusammenfassung}


In dem zweiten Teil des Experimentes wurde das Torsionspendel untersucht. Bei bekanntem Trägheitsmoment der Masse eignet es sich dazu das Schubmodul $G$ des Drahtes zu bestimmen und Informationen über das verwendete Material zu erhalten. So lies der gemessen Wert von $G=\SI{7.87+-0.09e10}{kg \per \second \squared  \metre}$ eine Stahllegierung vermuten.Bei bekanntem Aufbau und Schubmodul lässt sich alternativ das Trägheitsmoment eines Körpers um die, durch den Draht vorgegebene, Rotationsachse bestimmen. Durch unterschiedliche Aufhängungen der Hantlscheiben wurde der Einfluss des Abstandes zwischen Schwerpunkt und Rotationsachse untersucht und mithilfe des Steinerschen Satzes ausgewertet.