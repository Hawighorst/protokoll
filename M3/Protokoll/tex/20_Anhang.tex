

\subsection{Verwendete Gleichungen}\label{VGuD}
%und Definition der Variablen


%Zusammenhang zwischen Kreisfrequenz $\omega$ und Schwingungsdauer $T$:

%\begin{align}
%	T=\frac{2 \pi}{\omega} \pm \Delta t
%	\label{eq:T}
%\end{align} 


%Alle anderen Unsicherheiten sind gemäß Kapitel \ref{sec:einzeln} so klein, dass sie zu vernachlässigen sind. Es sei $\Delta t={ 0,006} {s}$.\\
%\frac{2 \pi}{\omega^2} \cdot\Delta \omega  \label{eq:T}


Standardunsicherheit der Rechteckverteilung u für die Intervallbreite a:
\begin{align}
	u=\frac{a}{2\sqrt{3}}\label{eq:sur}
	\end{align} 
	
	
Standardunsicherheit der Dreieckverteilung u: \begin{align}
	u=\frac{a}{2\sqrt{6}}\label{eq:sud}
\end{align}

Standardunsicherheit des Mittelwertes der Normalverteilung u für die Messwerte $x_i$ und den Mittelwert$\bar{x}$:
\begin{align}
	u(\bar{x})=  t_p  \sqrt{  \frac{\sum_{i=1}^{n}  (x_i-\bar{x})^2} {n (n-1)} }
	\label{eq:sunv}       
\end{align} 


Kominierte Standartunsicherheit der Messgröße $g(x_i)$

\begin{align}
	u(g(x_i))=   \sqrt{  \sum_{i=1}^{n} \left( \frac{\partial g}{\partial x_i} u(x_i) \right)^2  }
	\label{eq:kombsu}       
\end{align} 
\\
Elastizitätsmodul $E$:
\begin{align}
	E=\frac{F}{h_{\text{max}} I_q}\frac{L^3}{3}
	\label{eq:Elast}
\end{align}
\\
Flächenträgheitsmoment Kreis:
\begin{align}
	I_{\text{Kreis}}=\frac{\pi d^4}{64}
		\label{eq:TrägKreis}
\end{align}
\\
Flächenträgheitsmodul Rechteck:
\begin{align}		
	I_{\text{Rechteck}}= \frac{bc^3}{12}
	\label{eq:TrägRecht}
\end{align}
b senkrecht zur Biegungsebene, c waagerecht zu Biegungsebene.
\\
\\
Kraft:
\begin{align}
	F=10 \cdot a \cdot m
	\label{eq:Kraft}
\end{align}
	a ist die Steigung entnommen aus den Abbildungen \ref{fig:durchbiegungRund}, \ref{fig:durchbiegunEckig}
\\
Unsicherheit des Elastizitätsmoduls für Runde Stäbe:
\begin{align}
	u_E= E\cdot \sqrt{\left(\frac{u_a}{a}\right)^2+ \left(\frac{3\cdot u_L}{L}\right)^2 + \left( \frac{4 \cdot u_d}{d}\right)^2}
	\label{eq:UElastRund}
\end{align}	
\\
Unsicherheit des Elastizitätsmoduls für eckige Stäbe:
\begin{align}
	u_E= E \cdot \sqrt{ \left(\frac{u_a}{a}\right)^2 + \left( \frac{3 \cdot u_L}{L}\right)^2 + \left( \frac{u_b}{b}\right)^2 + \left(\frac{3 \cdot u_c}{c} \right)^2}
	\label{eq:UElastEckig}
\end{align}
	
	


