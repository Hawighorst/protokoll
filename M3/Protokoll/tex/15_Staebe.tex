\section{Stäbe}
In dieser Versuchsreihe wurde das Elastizitätsmodul von vier verschiedenen Stäben bestimmt.
\subsection{Methoden}
Die Durchführung dieses Experimentes erfolgte in zwei Abschnitten.
Im ersten Abschnitt wurde die Länge und die Dicke von vier verschiedenen Stäben bestimmt.
Die Dicke der Stäbe wurde mit einer Mikrometerschraube, die Länge  mit einem Maßband gemessen. Als Länge wird der Abstand zwischen Anfang der Einspannvorrichtung   und dem Aufhängepunkt des Gewichtes bezeichnet.
Da die Dicke der Stäbe mit der vierten Potenz in das Elastizitätsmodul eingeht, wurde sie an fünf Stellen dreifach gemessen.  Nach Abschluss dieser Messungen wurde im zweiten Abschnitt die Durchbiegung der Materialien in Abhängigkeit der angehängten Gewichte bestimmt. Zu diesem Zweck wurden die Stäbe auf der einen Seite eingespannt und auf der anderen Seite wurde ein Behälter eingehängt in den die Gewichte später hineingelegt wurden.
Nach jeder Messung wurde wieder die Ruhelage des Stabes bestimmt, damit bei Auftreten einer inelastischen Verbiegung diese erkannt werden konnte. Dies war bei keiner Messung der Fall. Zum Ablesen der Durchbiegung der Stäbe diente einer hinter den Stäben angebrachte Skala. Der Spiegel hinter der Skala ermöglichte ein weitgehend paralaxenfreies Ablesen.
Bei den Stäben handelte es sich um drei runde und einen rechteckigen Stab. Die Materialien wurden anhand der Farbe und dem Gewicht zunächst geschätzt.
Nach dieser Schätzung erhielt man einen runden Aluminium-, einen runden Stahl-, einen runden Messing- und einen eckigen Messingstab . Der rechteckige Stab wurde einmal Flachkant und einmal Hochkant eingespannt, um den Einfluss der Form eines Stabes auf die Durchbiegung und das Elastizitätsmodul zu untersuchen.

\subsection{Daten und Analyse und Diskussion}
Während des Experimentes wurde beobachtet, dass sich die Stäbe unterschiedlich gut elastisch verbiegen ließen (Stahl am besten und Aluminium am schlechtesten) und dass die Dicke eines Materials einen Einfluss auf die Biegsamkeit hat.

Bei  den 15 Messungen der Dicke der Stäbe wurden Schwankungen um ca. \SI{+-0,1}{mm} gemessen. Dies ist darauf zurückzuführen, dass die Mikrometerschraube von Hand festgezogen wurde und dem entsprechend nicht immer mit der gleichen Kraft angezogen wurde.
Die genauen Messungen sind in dem Laborbuch zu finden. Die Unsicherheiten der Dicke ergab sich nach den Gleichungen \ref{eq:sud}, \ref{eq:sunv} und \ref{eq:kombsu} aus der statistischen Unsicherheit und einer Ablesegenauigkeit von $\SI{\pm 2.5e-6}{m}$.   
Die Länge der Stäbe wurde mit einer Genauigkeit von \SI{+-1}{mm} gemessen, womit sich dann auch nach Gleichung \ref{eq:sud} die Unsicherheit für die Länge ergibt. Diese Abschätzung wurde auch bei der Durchbiegung verwendet.

\subsubsection*{Durchbiegung}
In diesem Abschnitt wurde die Durchbiegung in Abhängigkeit vom Gewicht untersucht. Zu diesem Zweck wurde dieser Zusammenhang in den Abbildungen \ref{figdurchbiegungRund} bzw. \ref{fig:durchbiegungEckig}  für die runden Stäbe bzw. für den eckigen Stab dargestellt. Da ein proportionaler Zusammenhang besteht wurde eine Anpassung\footnote{Die Anpassung wurde durch "Gnuplot" mit dem Levenberg–Marquardt Algorithmus vorgenommen.  }  mit $f(x)=a \cdot x$ erstellt. Die Unsicherheit dieser Anpassung wurde aus Gnuplot übernommen und ist in der Tabelle \ref{tab:Ela} angegeben.
Man erkennt das die Elastizität von Aluminium am schwächsten und die von Stahl am höchsten ist. Dies entspricht auch der alltäglichen Beobachtung, dass Stahl biegsamer ist als Aluminium.
Betrachtet man den Einfluss der Form bzw. der Dicke auf die Elastizität so erkennt man in \ref{fig:durchbiegungEckig}, dass die Hochkant eingespannte Stange sich weniger stark biegt als die Flachkant eingespannte Stange. Auch dies spiegelt alltägliche Beobachtungen wider, dass dickere Materialien weniger biegsam sind als dünnere Materialien.

\subsection{Elastizitätsmodul}


Zur Bestimmung des Elastizitätsmoduls nach Gleichung \ref{eq:Elast} werden die Gleichungen \ref{eq:TrägKreis} bzw. \ref{eq:TrägRecht} und \ref{eq:Kraft} eingesetzt. Unter einsetzen der in Tabelle \ref{tab:Ela} angegebenen Werte erhält man die ebenfalls in der Tabelle zu sehenden Werte für das Elastizitätsmodul.
Vergleicht man diese Werte für das Elastizitätsmodul mit Literaturwerten\footnote{Entnommen aus "Physik: für Wissenschaftler und Ingenieure" von Paul A. Tipler und Gene Mosca in der 7. Ausgabe von 2014.} in Tabelle \ref{tab:ElaLit} so erkennt man, dass es sich bei der runden Messingstange vermutlich eher um Kupfer handelt und nicht um Messing. Korrigiert man diese Annahme, so weichen die Werte für das Elastizitätsmodul um \SI{1}{\percent} bis \SI{7}{\percent} von den Literaturwerten ab. Betrachtet man dann noch die Unsicherheiten, die nach Gleichung \ref{eq:UElastRund} bzw. \ref{eq:UElastEckig} berechnet wurden, so erkennt man, dass die Werte bis auf Aluminium in ihrer $2\Sigma$ Umgebung den Literaturwerte einschließen. Jedoch kommt hinzu, dass diese Werte ebenfalls experimentell bestimmt wurden und von Quelle zu Quelle (Siehe zweite Quelle: Gerthsen Physik, Vogel 1977)\footnote{Hier ist der Elastizitätsmodul von Aluminium mit $E_{Al}=\SI{7,2e10}{N \per m^2}$ und von Duraaluminium mit $E_{Dura-Al}=\SI{7,7e10}{N \per m^2}$ angegeben. } schwanken. Weitere Faktoren die hinzukommen, sind mögliche andere Legierungenen (Messing) sowie verschiedene Herstellungsprozesse die Einfluss auf den Elastizitätsmodul haben. 


%\begin{table}[h]
%	\caption{Elastizitätsmodul E berechnet nach \ref{eq:Elast} mit allen dazu nötigen Werten}
%	\begin{tabular}{|c|c|c|c|c|c|c|}
%		\hline
%		& a $\left[ \frac{m}{g} \right]$& b[m]& c[m] & d[m] & L[m] & $E$ $\left[\frac{N}{m^2}  \%right]$ \\
%		\hline
%		Aluminium Rund & \SI{0,305+-0,006}{} & & & \SI{29,7+-0,002e-4}{} & \SI{0,2980+-0,0004}{} & \SI{7,492+-0,144e-10}{}\\
%		\hline
%		Messing Hochkant & \SI{0,041+-0,001}{} & \SI{20,0+-0,002e-4}{} & \SI{50+-0,002e-4}{} && \SI{0,2870+-0,0004}{} & \SI{9,429+-0,180e-10}{}\\
%		\hline
%		Messing Flachkant & \SI{0,264+-0,007}{} & \SI{50+-0,002e-4}{} & \SI{20,0+-0,002e-4}{} && \SI{0,2870+-0,0004}{} & \SI{9,313+-0,249e-10}{}\\
%		\hline
%		Messing Rund & \SI{0,202+-0,006}{} &&& \SI{29,6+-0,002e-4}{} & \SI{0,2950+-0,00045}{} & \SI{1,112+-0,034e-11}{}\\
%		\hline
%		Stahl Rund & \SI{0,114+-0,004}{} &&& \SI{29,7+-0,002e-4}{} & \SI{0,290+-0,0004}{} & \SI{1,815+-0,072e-11}{}\\
%		\hline
%	\end{tabular}
%\label{tab:Ela}
%\end{table}

\begin{table}[h]
	\caption{Elastizitätsmodul E berechnet nach \ref{eq:Elast} mit allen dazu nötigen Werten}
	{\scriptsize
	\begin{tabular}{|c|c|c|c|c|c|}
	\hline	
		& Aluminium & Messing  & Messing & 'Messing' & Stahl \\
		& Rund      & Hochkant & Flachkant & Rund  & Rund \\
	\hline
		a $\left[ \frac{m}{g} \right]$ &\SI{0,305+-0,006}{} & \SI{0,041+-0,001}{} & \SI{0,264+-0,007}{}  &  \SI{0,202+-0,006}{} & \SI{0,114+-0,004}{} \\
	\hline
		b[m] & & \SI{20,0+-0,002e-4}{} &  \SI{50+-0,002e-4}{} & & \\
	\hline
		 c[m] & & \SI{50+-0,002e-4}{} & \SI{20,0+-0,002e-4}{} & & \\
	\hline
		d[m] & \SI{29,7+-0,002e-4}{} & & & \SI{29,6+-0,002e-4}{} & \SI{29,7+-0,002e-4}{} \\
	\hline
		L[m] & 	\SI{0,2980+-0,0004}{} & \SI{0,2870+-0,0004}{} & \SI{0,2870+-0,0004}{} & \SI{0,2950+-0,0004}{} & \SI{0,290+-0,0004}{} \\
	\hline
		 $E$ $\left[\frac{N}{m^2}  \right]$ & \SI{7,492+-0,144e10}{} & \SI{9,429+-0,180e10}{} & \SI{9,313+-0,249e10}{} & \SI{1,112+-0,034e11}{} & \SI{1,815+-0,072e11}{}\\
	\hline
	\end{tabular}
}\label{tab:Ela}
\end{table}

\begin{table}[h]
	\caption{Literaturwerte für das Elastizitätsmodul \protect\footnotemark}
	\centering
	\begin{tabular}{|c|c|}
		\hline
		& Elastizitätsmodul E $\left[\frac{GN}{m^2}\right]$\\
		\hline
		Aluminium & 70 \\
		\hline
		Eisen & 190 \\
		\hline
		Kupfer & 110 \\
		\hline
		Messing & 90 \\
		\hline
	\end{tabular}
	\label{tab:ElaLit}
\end{table}
\footnotetext{Entnommen aus "Physik: für Wissenschaftler und Ingenieure" von Paul A. Tipler und Gene Mosca in der 7. Ausgabe von 2014.}

\begin{figure}[h]
	\centering
	\includegraphics[width=1\textwidth]{res/Rund.pdf}
	\caption{Durchbiegung der runden Stäbe in Abhängigkeit des Gewichts.}
	\label{figdurchbiegungRund}
\end{figure}

\begin{figure}[h]
	\centering
	\includegraphics[width=1\textwidth]{res/Eckig.pdf}
	\caption{Durchbiegung des in zwei verschiedenen Ausrichtungen eingespannten rechteckigen Stabes in Abhängigkeit des Gewichts.}
	\label{fig:durchbiegungEckig}
\end{figure}



%Fitfunktion Unsicherheit der Fitfunktion Warum die von Gnuplot?