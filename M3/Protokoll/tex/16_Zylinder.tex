\section{Torsionsschwingung}


\subsection{Methoden}

\subsubsection*{Zylinder}




\subsubsection*{Hantel}





\subsection{Daten und Analyse}
% Gemessen wurden die Werte win in Tabelle ?? dargestellt.
 
 
 %Beschreibung unsicherheiten entstehung
 %Rhdaten T L??
 
 
 
\begin{table}[h]
\centering	
\caption{Messdaten des Torsionspendels mit Zylinder}
 \begin{tabular}{|l|c|}
 
 	\hline 
 Messgröße	& Messwert  \\ 
 	\hline 
 	Länge des Drahtes $L$& \SI{1.8150\pm 0.0004 } {m} \\ 
 	\hline 
 	Masse des Zylinders $m_z$& \SI{2.648}{kg} \\ 
 	\hline 
 	Radius des Zylinders $R_z$ & \SI{0.0735 \pm 0.0004}{m}  \\ 
 	\hline 
 	Radius des Drahtes & \SI{2.50+-0.002 e-4} {m} \\ 
 	\hline 
 	Gemittelte Schwingungsdauer $T$&\SI{97.74+-0.11}{s}  \\ 
 	\hline 
 \end{tabular} 

\end{table} 




%beobachtung









\subsection{Diskussion}


%Wdh Ergebnisse , Tabelle für vgl. Einordnung