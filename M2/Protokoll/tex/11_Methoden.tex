\section{Methoden}
Betrachtet wurden zwei mithilfe einer Feder gekoppelten Stabpendel.
Zunächst wurde die Pendellängen der beiden Pendel so eingestellt dass sie einzeln die gleichen Schwingungsdauern aufwiesen. Betrachtet man die Differentialgleichungen, welche das System der gekoppelten Pendel beschreiben, so besteht die allgemeine Lösung aus der Überlagerung von zwei Eigenschwingungen. Diese sind die gleichsinnige bzw. gegensinnige Eigenschwingung. Bei der gleichsinnigen Eigenschwingung schwingen die beiden Pendel synchron, während bei der gegensinnigen Eigenschwingung die Pendel einen Phasensprung von $\pi$ aufweisen und folglich spiegelverkehrt laufen. Beide Lösungen zeichnen sich dadurch aus, dass die Feder keine Energie zwischen den Pendeln überträgt.
 Um diese Lösungen im Experiment beobachten zu können, wurden die Pendel zu ihren Eigenschwingungen angeregt. \\
 
 
Bei der gleichsinnigen Schwingung geschah dies dadurch, dass beide Pendel um den selben Winkel in die selbe Richtung ausgelenkt und zeitgleich losgelassen werden.
Bei der gegensinnigen Schwingung geschah dies dadurch, dass beide Pendel um den selben Winkel in verschiedene Richtungen, hier beide zur Mitte, ausgelenkt und zeitgleich losgelassen wurden.
Anschließend sollte die Überlagerung der beiden Lösungen, die Schwebung beobachtet werden. hierzu wurde nur eines der beiden Pendel ausgelenkt.
 
Die beschriebenen Versuche wurden jeweils mit zwei verschieden harten Federn durchgeführt. Die kupferfarbene Feder war augenscheinlich weicher als die Metallfarbene. Der besseren Lesbarkeit werden die Federn im folgenden als Kupfer- bzw. Edelstahlfeder bezeichnet. Des weiteren wurde der Kopplungsgrad statisch und aus den gewonnenen Daten über die relative Frequenzaufspaltung berechnet. Die Schwingungszustände wurden mit denen des Doppelpendels verglichen.

 
















