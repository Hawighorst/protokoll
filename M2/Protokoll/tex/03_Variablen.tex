% Autor: Simon May
% Datum: 2016-10-13
% Der Befehl \newcommand kann auch benutzt werden, um „Variablen“ zu definieren:

% Nummer laut Praktikumsheft:
\newcommand*{\varNum}{E5}
% Name laut Praktikumsheft:
\newcommand*{\varName}{Drehpendel nach Pohl}
% Datum der Durchführung (Format: JJJJ-MM-TT):
\newcommand*{\varDatum}{2017-11-22}
% Autoren des Protokolls:
\newcommand*{\varAutor}{Hauke Hawighorst, Jörn Sievneck}
% Nummer der eigenen Gruppe:
\newcommand*{\varGruppe}{Gruppe 9}
% E-Mail-Adressen der Autoren (kommagetrennt ohne Leerzeichen!):
\newcommand{\varEmail}{h.hawighorst@uni-muenster.de,j\_siev11@uni-muenster.de}
%betreuer Name
\newcommand{\varBetreuer}{\normalsize betreut von \\ Martin Körsgen  }
% E-Mail-Adresse anzeigen (true/false):
\newcommand*{\varZeigeEmail}{true}
% Kopfzeile anzeigen (true/false):
\newcommand*{\varZeigeKopfzeile}{true}
% Inhaltsverzeichnis anzeigen (true/false):
\newcommand*{\varZeigeInhaltsverzeichnis}{true}
% Literaturverzeichnis anzeigen (true/false):
\newcommand*{\varZeigeLiteraturverzeichnis}{true}

