\subsection{Einfluss eines Magneten auf eine Aluminiumplatte und auf einen Aluminiumkamm }
Bei diesem Experiment wurde die Auswirkung eines Experimentes auf zwei verschiedene Aluminiumplatten 
untersucht. Bei der eine Platte handelte es sich um einen Aluminiumkamm während die andere Platte vollständig war.
In der ersten Versuchsreihe wurde ein Magnet langsam an die platten angenähert und dann wieder Langsam entfernt.
Bei der zweiten Versuchsreihe wurde der Magnet mehrfach schnell an die Platten angenähert und dann wider entfernt.
Die Auswirkungen auf die Platte und auf dem Kamm unterschieden sich stark.
Beim Kamm hatte das annähern des Magnets kaum Auswirkungen. 
Einen Effekt hatte der Magnet nur beim langsamen annähern an den Kamm. Aber auch diese Auswirkung ist nur minimal denn der Kamm wird von dem Magneten kaum angezogen.
Die Platte verhielt sich jedoch anders als der Kamm, denn beim schnellen annähern des Magneten an die Platte wurde die Platte abgestoßen.
Der Effekt der schon beim Kamm auftrat also das die Platte beim langsamen annähern des Magneten die Platte angezogen wird nur das der Effekt hier deutlich stärker ist als beim Kamm.
Dieser unterschied ist darauf zurückzuführen das der Kamm im Gegensatz zu der Platte Unterbrechungen in der Oberfläche besitzt. Das führt dazu dass die Kreisströme die auf die Oberfläche induziert werden nur sehr schwach sind.

