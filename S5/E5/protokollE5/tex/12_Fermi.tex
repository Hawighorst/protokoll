\subsection{Einfluss eines Magneten auf Wasser - Eine Fermi-Abschätzung. }
Bei diesem Experiment ging es darum den Einfluss eines Magneten auf Wasser zu untersuchen.
Zu diesem Zweck wurde ein Versuchsaufbau analog zu \cref{fig:aufbau} erstellt. Anhand des Lasers der auf der Wasseroberfläche reflektiert wurde konnte man erkennen ob der Magnet einen Einfluss auf das Wasser hatte. bei der durchführung war zu beobachten das der Laser sich erst nach oben und dann nach unten bewegte. Diese verhalten lässt sich darauf zurückführen das sich über dem Magneten eine einbuchtung bildete die in diesem Fall durch ein wie in der Abbildung zusehen durch ein nach unten gerichtetes dreieck aproximiert wurde. Das verhalten des Wassers lässt sich dadurch erklären das es sich bei Wasser um einen Diamagnitischen Stoff handelt der Von einem Magnetischen Kraftfeld abgestoßen wird. Dies wiederspricht scheibar erstmal der Beobachtung das sich über dem Magneten eine Kuhle gebildet hat. Da Wasser jedoch flüssig sit fließt es zu alle seiten weg vom Magneten und es belibt direkt über dem Magneten eine Einbuchtung in der OPberfläcjhe zurück.
Mithilfe einer Fermi abschätzung sollte nun abgeschätzt werden wie stark das Wasser von dem Magneten Beinflusst wird. Dazu wurde die Höhe des Dreiecks bestimmt.