%16
\section{Fazit}
Beim durchführen der Experimente wurden verschiedenes Magnetisches Verhalten beobachtet.
Die untersuchten Stoffe waren Paramagnetisch (Aluminium, Graphit) und Diamagnetisch (Glas, Wasser),
jedoch konnte auch bei Aluminium Diamagnetisches Verhalten beobachtet werden sobald das Magnetfeld stark genug war.
Generell wurde beobachtet das Diamagnetische Effekte nur in Starken Magnetfeldern auftreten. Des weiteren wurde der Einfluss der 
Oberfläche eines Objektes auf seine Magnetischen Eigenschaften untersucht und es stellte sich heraus das unterbrochene Oberflächen die Stärke der Auftretenden Effekte stark reduzieren.
Ein weiterer Aspekt der untersucht wurde war das man die Entscheidung ob ein Material Paramagnetisch, Diamagnetisch oder Ferromagnetisch nicht dadurch bestimmt das man versucht das verhalten des Stoffes Optisch zu untersuchen,
sondern stattdessen die Eigenschaften über die Volumensuszebilität bestimmt wie es in \cref{Kap:Vol} gemacht wurde.
Eine Möglichkeit Magnetische Effekte die Man mit bloßem Auge nicht erkennen kann sichtbar zu machen wurde in \cref{Kap:Fermi} vorgestellt. Die dort vorgestellte Methode ist jedoch sehr ungenau da nur mit Schätzwerten gerechnet wird,
jedoch eignet sich der Wert recht gut um eine Stoff grob zu Klassifizieren.