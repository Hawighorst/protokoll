\subsection{Verhalten eines Magneten beim Fall durch ein Aluminiumrohr mit und ohne Schlitz.}

Bei diesem Experiment wurde die Fallgeschwindigkeit eines Magneten durch 
zwei verschieden Aluminiumrohren untersucht.
Das eine Rohr hat einen Schlitz in der Oberfläche sodass es eine Unterbrechung der Oberfläche gibt während das andere Rohr diese nicht besitzt.
Um das Verhalten zu untersuchen wurde ein Magnet in dem jeweiligen Rohr fallengelassen.
Bei der Durchführung des Experimentes wurde beobachtet das der Magnet durch das eingeschlitzte Rohr zwar Langsamer fällt als es im Normalfall 
der Fall wäre ist jedoch deutlich schneller als beim Fall durch das nicht eingeschlitzte Rohr.
Dies liegt daran das beim Fall durch das Rohr durch die Magneten Kreisströme auf die Oberfläche induziert werden.
Diese Kreisströme treten jedoch beim eingeschlitzten Rohr nur sehr schwach auf da dort keine 'echten' Kreisströme auftreten können( wegen der durchbrochenen Oberfläche).
Durch die Wechselwirkung zwischen den Kreisströmen und dem Magneten wird der Fall stark verlangsamt. 
Ein solches verhalten tritt bei allen Paramagnetischen Metallen auf, jedoch nicht immer mit der gleichen Stärke. 

