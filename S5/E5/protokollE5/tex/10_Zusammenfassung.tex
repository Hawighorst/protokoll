\section{Kurzfassung}
In dieser Experiment reihe ging es darum auf Experimenteller Basis die verschiedenen Magnetischen Effekte zu untersuchen,
und zu klassifizieren. Zu diesem Zweck wurden verschiedene Materialien in eine Magnetfeld gebracht und die Auswirkungen untersucht. 
Ein weiterer schritt war es die Auswirkung der Oberfläche eines Stoffes auf die Magnetisierung zu untersuchen.
Die drei am häufigsten anzutreffen Magnetischen Effekte sind:
\begin{itemize}
	\item Ferromagnetismus: Diese Stoffe werden in einem Magnetfeld stark angezogen.
	Dieser Effekt tritt auch außerhalb von Magnetfeldern auf.
	\item Paramagnetismus: Dieser effekt tritt nur in Magnetfeldern auf. Paramagnetische Stoffe werden in ein Magnetfeld hineingezogen, jedoch ist dieser Effekt nicht besonders stark.
	\item Diamagnetismus: Diamagnetismus tritt in allen Stoffen auf, jedoch ist dieser Effekt so schwach das er von anderen Magnetischen Effekten überlagert wird es sei denn das Magnetfeld ist sehr stark.
	In einem Magnetfeld werden solche Stoffe hinausgedrängt.
\end{itemize}
In den Experiment reihen wurde der Einfluss eines Magneten auf Wasser, Aluminium, Graphit und Glas untersucht sowie der Einfluss den die Oberfläche eines Objektes(Einer Aluminiumplatte und eines Aluminiumkammes) auf
die auftretenden Magnetischen Effekte hat.